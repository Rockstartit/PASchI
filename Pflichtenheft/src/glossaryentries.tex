\newcommand{\glossaryentry}[2] {
    \newglossaryentry{#1}{
        name={#1},
        description={#2}
    }
}

\glossaryentry{PWA}{Progressive Web App. Technologie, die es ermöglicht, Webapplikationen auf Geräten mit unterstützendem \Gls{Browser} installieren zu können, sodass diese auch offline funktionieren (mit einem eigenen \Gls{Cache}, der mithilfe von Service-Workern mit dem Server synchronisiert werden kann)}
\glossaryentry{Frontend}{Das Frontend ist in einer Schichteneinteilung der Teil des Programms, der näher am Benutzer ist}
\glossaryentry{Backend}{Das Backend ist in einer Schichteneinteilung der Teil des Programms, der näher am System ist}
\glossaryentry{Browser}{Anwendung, die die Darstellung von Seiten im World Wide Web ermöglicht}
\glossaryentry{Server}{Der Begriff Server beschreibt einen großen und leistungsstarken Computer, auf dem ein Programm läuft, dass anderen Programmen im Rahmen eines Client-Server Modells Dienste zentral zur Verfügung stellt}
\glossaryentry{TypeScript}{TypeScript ist eine Programmiersprache zur Implementierung von Logik auf Internetseiten}
\glossaryentry{HTML5}{Hypertext Markup Language 5 ist die fünfte Version einer Computersprache zur Auszeichnung und Gestaltung von Texten und anderen Inhalten. Sie enthält keine Logik}
\glossaryentry{Framework}{Ein Framework ist ein Rahmen für eine Anwendung. Es ist selbst noch kein vollständiges Programm, sondern vielmehr ein Werkzeug für den Programmierer, um ein Programm einfacher und schneller entwickeln zu können}
\glossaryentry{VM}{Eine virtuelle Maschine ist ein Softwareprogramm, die einen Computer simuliert}
\glossaryentry{Java}{Java ist eine objektorientierte Programmiersprache, die geräteunabhängig ausgeführt werden kann}
\glossaryentry{GUI}{Die Grafische Benutzeroberfläche bezeichnet eine \Gls{Schnittstelle} zwischen dem Benutzer und dem jeweiligen Gerät}
\glossaryentry{Account}{Ein Benutzerkonto (engl. Account) ist eine Zugangsberechtigung zu einem System}
\glossaryentry{Dashboard}{Das Dashboard ist eine grafische Oberfläche zur übersichtlichen Visualisierung von Daten}
\glossaryentry{undo}{Die undo-Funktion bezeichnet eine Funktion in einem Programm, mit der die letzte ausgeführte Aktion rückgängig gemacht werden kann}
\glossaryentry{Themes}{Ein Theme ist eine vorgefertigte Zusammenstellung grafischer Elemente, bspw. von Farben, Schriftart oder Schriftgröße}
\glossaryentry{System Usability Scale}{Der System Usability Scale basiert auf einem einfachen Fragebogen, der technologieunabhängig das Maß der Gebrauchsfertigkeit eines Systems bewertet. Durch Auswerten der Fragebögen wird ein Score bestimmt, der zwischen null und 100 liegen kann. Je höher der Score, desto besser ist das System zu benutzen. 68 Punkte entsprechen einem guten Ergebnis}
\glossaryentry{Live}{Live bedeutet in Echtzeit}
\glossaryentry{Desktop}{Mit Desktop ist im Allgemeinen ein Computer gemeint. Hier spielt vor allem die meist größere Ansicht eine Rolle}
\glossaryentry{Chromium}{Chromium ist ein \Gls{Browser}, der als Grundlage vieler anderer Browser gilt. Es ist von dem Unternehmen Google}
\glossaryentry{Synchronisation}{Als Synchronisation bezeichnet man den Abgleich von Daten via Internetverbindung}
\glossaryentry{Android}{Ein Betriebssystem für Smartphones. Siehe \href{https://source.android.com/ }{Android Open Source Project}}
\glossaryentry{iOS}{Ein Betriebssystem für Smartphones des Unternehmens Apple}
\glossaryentry{Display}{Das Display ist eine Technik zum Anzeigen von digitalen Informationen}
\glossaryentry{Vue.js}{Spring-Boot ist ein \Gls{Frontend}-\Gls{Framework} für die Programmiersprachen \Gls{TypeScript} und \Gls{Javascript}}
\glossaryentry{Spring-Boot}{Spring-Boot ist ein \Gls{Backend}-\Gls{Framework} für die Programmiersprachen \Gls{Java} und \Gls{Kotlin}}
\glossaryentry{Datenbank}{Eine Datenbank ist ein System, welches Daten speichern, verwalten und wiedergeben kann}
\glossaryentry{Rest}{Representational State Transfer ist ein Architekturstil für die Erstellung von \Gls{API}s. Rest dient zur Maschine-zur-Maschine-Kommunikation}
\glossaryentry{API}{Eine API ist eine \Gls{Schnittstelle}, die von einem Softwaresystem zur Verfügung gestellt wird und anderen Programmen eine Anbindung an das System zugeben}
\newglossaryentry{Schnittstelle}{
name={Schnittstelle},
description={Die Schnittstelle ist ein Programmteil, das zur Kommunikation dient},
text={schnittstelle}}
\glossaryentry{Kotlin}{Kotlin ist eine Programmiersprache mit häufigem Einsatz in der Entwicklung von \Gls{Android}- und \gls{iOS}-Apps. Sie läuft mit der Java Virtuell Machine}
\glossaryentry{Javascript}{Javascript ist eine Programmiersprache, um hauptsächlich Logik auf Internetseiten zu implementieren}
\newglossaryentry{gehasht}{
name=Hashing,
description=Passwörter werden mittels eines Passwort-Hashing-Verfahrens in eine festgelegte Codefolge mit zufälligen Zahlen und Buchstaben umgewandelt,
text=gehasht}
\glossaryentry{SSL}{Secure Sockets Layer (SSL) ist ein Protokoll zur Verschlüsselung der Kommunikation im Internet}
\glossaryentry{HTTP}{Hypertext Transfer Protocol (HTTP) ist ein Protokoll zur Übertragung von Daten in einem Rechnernetz. Es wird insbesondere verwendet, um Webseiten aus dem World Wide Web in den \Gls{Browser} zu laden}
\glossaryentry{MySQL}{MySQL ist ein Open-Source Datenbankverwaltungssystem}
\glossaryentry{Open-Source}{Als Open-Source bezeichnet man Programme/Programmquelltext der von Dritten eingelesen, genutzt und geändert werden}
\glossaryentry{Swipe}{Das Wischen (engl. swipe) bezeichnet die "Wisch"-Bewegung bei der Bedienung eines Touchscreens}
\glossaryentry{Touchscreen}{Der Touchscreen ist ein \Gls{Display}, welches ein zusätzliches Eingabegerät für Berührungen hat}
\glossaryentry{Login}{Der Login bezeichnet das Anmelden an einem digitalen System}
\glossaryentry{Administrator}{Verwalter des Produktes mit erweiterten Funktionen}
\glossaryentry{Safari}{Safari ist ein \Gls{Browser}, der auf dem Betriebssystem \gls{iOS} vorinstalliert ist}
\glossaryentry{Cache}{Ein Cache ist ein schneller Pufferspeicher eines Computers}

\makenoidxglossaries