\renewcommand{\figurename}{Abb.}
\section{Benutzeroberfläche / Schnittstellen}
In diesem Kapitel wird die Benutzeroberfläche erläutert und auf die beispielhafte Darstellung einzelner Funktionen eingegangen. Die benutzten Bilder sollen dabei helfen, sich das Produkt vorzustellen. Das fertige Produkt kann von dieser Darstellung abweichen.
\subsection{Benutzeroberfläche}
Das Produkt ist eine \Gls{PWA} mit einer mobilen und einer \Gls{Desktop}-Ansicht. Da das Produkt sowohl auf \Gls{Desktop}rechnern als auch auf Smartphones laufen kann, gibt es je nach Gerät verschiedene Steuerungsmöglichkeiten: Touch bzw. Maus und Tastatur. Das Produkt soll dabei einfach, übersichtlich und flüssig in der Bedienung sein (siehe \Gls{System Usability Scale}).
Es gibt eine Startseite, von dem aus mithilfe von Buttons verschiedene Funktionen gewählt werden können.

\subsection{Login}
Der Benutzer muss sich nach dem Öffnen der Anwendung registrieren bzw. anmelden, falls er abgemeldet ist. Nach erfolgreichem \Gls{Login} gelangt der Nutzer auf die Startseite. Für die Registrierung ist die Eingabe eines Benutzernamens und das Festlegen eines Passworts nötig. Wenn ein Nutzer beim Öffnen der Anwendung bereits eingeloggt ist, gelangt dieser direkt auf die Startseite.

\subsection{Startseite}
Auf der Startseite wird der Benutzername des Nutzers angezeigt. Es werden Buttons für die Nutzung einiger Funktionen des Produkts und zur Navigation in Untermenüs (siehe \autoref{img: mobile Startseite}) angezeigt. In der Desktop-Ansicht können durch die größere Bildschirmfläche mehr Funktionen dargestellt werden (siehe \autoref{img: Desktop Startseite}).
\begin{figure}[H]
\begin{minipage}{.5\textwidth}
    \centering
    \includesvg[inkscapelatex=false,width=0.4\textwidth]{graphics/MainMenuPhone.svg}
    \captionof{figure}{Mobile Startseite}
    \label{img: mobile Startseite}
\end{minipage}
\begin{minipage}{.5\textwidth}
 \centering
    \includesvg[inkscapelatex=false,width=0.8\textwidth]{graphics/DesktopDashboard.svg}
    \captionof{figure}{Desktop Startseite}
    \label{img: Desktop Startseite}
\end{minipage}%
\end{figure}

\subsection{Schülerbibliothek}
\autoref{img: Schülerliste} zeigt, wie dieses Menü aussehen könnte. Es wird eine Liste der Schüler angezeigt. Diese können durch eine Suchleiste gefunden werden und dann bearbeitet (siehe \autoref{img: Schüler bearbeiten}) oder gelöscht werden. Zudem können neue Schüler angelegt werden (siehe \autoref{img: Schüler anlegen}).
\begin{figure}[H]
\begin{minipage}{.5\textwidth}
    \centering
    \includesvg[inkscapelatex=false,width=0.4\textwidth]{graphics/StudentListMobile.svg}
    \captionof{figure}{Schülerliste}
    \label{img: Schülerliste}
\end{minipage}
\begin{minipage}{.5\textwidth}
 \centering
    \includesvg[inkscapelatex=false,width=0.4\textwidth]{graphics/EditStudentPhone.svg}
    \captionof{figure}{Schüler bearbeiten}
    \label{img: Schüler bearbeiten}
\end{minipage}%
\end{figure}
\begin{figure}[H]
    \centering
    \includesvg[inkscapelatex=false,width=0.2\textwidth]{graphics/CreateStudentPhone.svg}
    \captionof{figure}{Schüler anlegen}
    \label{img: Schüler anlegen}
\end{figure}

\subsection{Sitzung}
Während einer Sitzung erhält man einen Überblick über die Schüler in einem Kurs (siehe \autoref{img: mobile Schülerauswahl}). In der Desktop-Ansicht kann eine Sitzordnung angezeigt werden (siehe \autoref{img: Desktop Schülerauswahl}). Es soll nach der Auswahl eines Schülers möglich sein, ein Interaktionsziel auszuwählen (siehe \autoref{img: Zielschüler auswählen}) und die Interaktion zu kategorisieren (siehe \autoref{img: mobile Interaktionbewertung} und \autoref{img: Desktop Interaktionsbewertung}).

\begin{figure}[H]
\begin{minipage}{.5\textwidth}
    \centering
    \includesvg[inkscapelatex=false,width=0.4\textwidth]{graphics/SelectStudentPhone.svg}
    \captionof{figure}{Mobile Schülerauswahl}
    \label{img: mobile Schülerauswahl}
\end{minipage}
\begin{minipage}{.5\textwidth}
 \centering
    \includesvg[inkscapelatex=false,width=0.4\textwidth]{graphics/SelectTargetPhone.svg}
    \captionof{figure}{Zielschüler auswählen}
    \label{img: Zielschüler auswählen}
\end{minipage}%
\end{figure}

\begin{figure}[H]
    \centering
    \includesvg[inkscapelatex=false,width=0.5\textwidth]{graphics/TabletSelectStudent.svg}
    \captionof{figure}{Desktop Schülerauswahl}
    \label{img: Desktop Schülerauswahl}
\end{figure}

\begin{figure}[H]
\begin{minipage}{.5\textwidth}
    \centering
    \includesvg[inkscapelatex=false,width=0.4\textwidth]{graphics/RateInteractionPhone.svg}
    \captionof{figure}{Mobile Interaktionbewertung}
    \label{img: mobile Interaktionbewertung}
\end{minipage}
\begin{minipage}{.5\textwidth}
    \centering
    \includesvg[inkscapelatex=false,width=1\textwidth]{graphics/TabletRateInteraction.svg}
    \captionof{figure}{Desktop Interaktionsbewertung}
    \label{img: Desktop Interaktionsbewertung}
\end{minipage}%
\end{figure}

\subsection{Raumbearbeitung}
Befindet man sich in der Desktop-Ansicht, können Schüler in einem Raum hinzugefügt und entfernt werden (siehe \autoref{img: Klassenraum bearbeiten}). Man kann mehrere Sitzordnungen für einen Kurs anlegen.
\begin{figure}[H]
    \centering
    \includesvg[inkscapelatex=false,width=0.6\textwidth]{graphics/EditClassroom.svg}
    \caption{Klassenraum bearbeiten}
    \label{img: Klassenraum bearbeiten}
\end{figure}

\subsection{Statistiken}
Hier können die gegebenen Statistiken für einen Kurs angesehen werden. Eine mögliche Statistik in der Desktop-Ansicht zeigt \autoref{img: Interaktionskarte anzeigen}.
\begin{figure}[H]
    \centering
    \includesvg[inkscapelatex=false,width=0.6\textwidth]{graphics/TabletViewInteractionMap.svg}
    \caption{Interaktionskarte anzeigen}
    \label{img: Interaktionskarte anzeigen}
\end{figure}