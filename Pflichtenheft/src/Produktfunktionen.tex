\section{Produktfunktionen}
In diesem Kapitel werden die Funktion des Produkts genauer beschrieben. Bei der Produktfunktionalität wird zwischen den Basis-Funktionen und den Erweiterten Funktionen unterschieden. Die Basis-Funktionen stellen die grundlegenden Funktionalitäten des Produkts dar. Sie sind notwendig, um die Musskriterien\textsuperscript{\ref{list:Musskriterien}} zu erfüllen. Die erweiterten Funktionen ergänzen die Anwendung entsprechend der Kannkriterien\textsuperscript{\ref{list:Kannkriterien}}.

\subsection{Basis-Funktionen}
Die folgende Tabelle gibt einen Überblick über die Basis-Funktionen. Alle Funktionen werden in \ref{section:beschreibungen} genauer beschrieben.\\

\begin{table}[!ht]
    \centering
        \begin{tabular}{p{1cm}|p{9.5cm}|p{3cm}}
                \textbf{\sffamily{Nr.}} & \textbf{\sffamily{Funktion}} & \textbf{\sffamily{Kriterium}}\textsuperscript{\ref{list:Musskriterien}}\\
            \hline
            \hline
                $\langle$\textit F010$\rangle$ & Anzeigen des \Gls{Login}-Bildschirms & $\langle$\textit{MK}1$\rangle$\\
            \hline
                $\langle$\textit F020$\rangle$ & \Gls{Account} anlegen und einloggen & $\langle$\textit{MK}1$\rangle$,$\langle$\textit{MK}12$\rangle$\\
            \hline
                $\langle$\textit F030$\rangle$ & Schüler anzeigen & $\langle$\textit{MK}2$\rangle$\\
            \hline
                $\langle$\textit F040$\rangle$ & Neuen Schüler anlegen & $\langle$\textit{MK}2$\rangle$\\
            \hline
                $\langle$\textit F050$\rangle$ & Schülerstatistiken ansehen & $\langle$\textit{MK}2$\rangle$,$\langle$\textit{MK}10$\rangle$,\newline$\langle$\textit{MK}11$\rangle$,$\langle$\textit{MK}12$\rangle$\\
            \hline
                $\langle$\textit F060$\rangle$ & Kurse anzeigen & $\langle$\textit{MK}3$\rangle$\\
            \hline
                $\langle$\textit F070$\rangle$ & Neuen Kurs anlegen & $\langle$\textit{MK}3$\rangle$\\
            \hline
                $\langle$\textit F080$\rangle$ & Kursstatistiken ansehen & $\langle$\textit{MK}3$\rangle$,$\langle$\textit{MK}10$\rangle$,\newline$\langle$\textit{MK}11$\rangle$,$\langle$\textit{MK}12$\rangle$\\
            \hline
                $\langle$\textit F090$\rangle$ & Schüler in einem Kurs anzeigen & $\langle$\textit{MK}3$\rangle$,$\langle$\textit{MK}4$\rangle$\\
            \hline
                $\langle$\textit F100$\rangle$ & Schüler zu einem Kurs hinzufügen & $\langle$\textit{MK}3$\rangle$,$\langle$\textit{MK}4$\rangle$\\
            \hline
                $\langle$\textit F110$\rangle$ & Schüler aus einem Kurs entfernen & $\langle$\textit{MK}3$\rangle$,$\langle$\textit{MK}4$\rangle$\\
            \hline
                $\langle$\textit F120$\rangle$ & Neue Sitzung erstellen & $\langle$\textit{MK}5$\rangle$,$\langle$\textit{MK}6$\rangle$\\
            \hline
                $\langle$\textit F130$\rangle$ & Interaktion aufzeichnen & $\langle$\textit{MK}5$\rangle$,$\langle$\textit{MK}6$\rangle$,\newline$\langle$\textit{MK}7$\rangle$,$\langle$\textit{MK}8$\rangle$\\
            \hline
                $\langle$\textit F140$\rangle$ & Kategorie für Interaktion auswählen & $\langle$\textit{MK}5$\rangle$,$\langle$\textit{MK}9$\rangle$\\
            \hline
                $\langle$\textit F150$\rangle$ & Interaktionskarten speichern & $\langle$\textit{MK}5$\rangle$,$\langle$\textit{MK}12$\rangle$\\
            \hline
                $\langle$\textit F160$\rangle$ & Interaktionskarten ansehen & $\langle$\textit{MK}5$\rangle$,$\langle$\textit{MK}6$\rangle$,\newline$\langle$\textit{MK}7$\rangle$,$\langle$\textit{MK}12$\rangle$\\
            \hline
                $\langle$\textit F170$\rangle$ & Sitzungsstatistiken ansehen & $\langle$\textit{MK}5$\rangle$,$\langle$\textit{MK}10$\rangle$,\newline$\langle$\textit{MK}12$\rangle$\\
            \hline
                $\langle$\textit F180$\rangle$ & Interaktionskarten exportieren & $\langle$\textit{MK}5$\rangle$,$\langle$\textit{MK}13$\rangle$\\
            \hline
        \end{tabular}
    \caption{Basis-Funktionen}
    \label{table:Basis-Funktionen}
\end{table}

\newpage
\subsection{Erweiterte Funktionen}
Die folgende Tabelle gibt einen Überblick über die erweiterten Funktionen. Alle Funktionen werden in \ref{section:beschreibungen} genauer beschrieben.\\

\begin{table}[!ht]
    \centering
        \begin{tabular}{p{1cm}|p{9.5cm}|p{3cm}}
                \textbf{\sffamily{Nr.}} & \textbf{\sffamily{Funktion}} & \textbf{\sffamily{Kriterium}}\textsuperscript{\ref{list:Kannkriterien}}\\
            \hline
            \hline
                $\langle$F190$\rangle$ & Schüler teilen & $\langle$\textit{KK}1$\rangle$,$\langle$\textit{MK}1$\rangle$,\newline$\langle$\textit{MK}2$\rangle$\\
            \hline
                $\langle$F200$\rangle$ & Kurse teilen  & $\langle$\textit{KK}1$\rangle$,$\langle$\textit{MK}1$\rangle$,\newline$\langle$\textit{MK}3$\rangle$\\
            \hline
                $\langle$F210$\rangle$ & Sitzordnung für Kurse anlegen & $\langle$\textit{KK}2$\rangle$,$\langle$\textit{KK}3$\rangle$,\newline$\langle$\textit{MK}3$\rangle$\\
            \hline
                $\langle$F220$\rangle$ & Unterrichtsfach zu einem Kurs hinzufügen &  $\langle$\textit{KK}4$\rangle$,$\langle$\textit{MK}3$\rangle$\\
            \hline
                $\langle$F230$\rangle$ & Interaktion rückgängig machen mit \Gls{undo}  &  $\langle$\textit{KK}5$\rangle$,$\langle$\textit{MK}5$\rangle$\\
            \hline
                $\langle$F240$\rangle$ & Qualität bei Interaktion angeben &  $\langle$\textit{KK}4$\rangle$,$\langle$\textit{MK}5$\rangle$\\
            \hline
                $\langle$F250$\rangle$ & Eigene Kategorie für Interaktionen erstellen &  $\langle$\textit{KK}7$\rangle$,$\langle$\textit{MK}5$\rangle$,\newline$\langle$\textit{MK}9$\rangle$\\
            \hline
        \end{tabular}
    \caption{Erweiterte Funktionen}
    \label{table:Erweiterte Funktionen}
\end{table}

\newenvironment{beschreibung}{%
  \parskip6pt \parindent0pt \raggedright
  \def\lititem{\hangindent=0.95cm \hangafter1}}{%
  \par\ignorespaces}

\onehalfspacing
\subsection{Funktionsbeschreibungen}
\label{section:beschreibungen}

\textbf{\large \sffamily {Anzeigen des \Gls{Login}-Bildschirms}} $\langle$\textit F10$\rangle$

\begin{beschreibung}
    \lititem{\textbf{\sffamily Anwendungsfall:} Der Nutzer öffnet die Anwendung.\\}
    \lititem{\textbf{\sffamily Anforderung:} MK1\\}
    \lititem{\textbf{\sffamily Ziel:} Gibt dem Nutzer die Möglichkeit sich anzumelden bzw. zu registrieren.\\}
    \lititem{\textbf{\sffamily Vorbedingung:} -\\}
    \lititem{\textbf{\sffamily Nachbedingung:} Der \Gls{Login}-Bildschirm wird angezeigt.\\}
    \lititem{\textbf{\sffamily Akteure:} Nutzer, Server\\}
    \lititem{\textbf{\sffamily Auslösendes Ereignis:} Die Anwendung wird geöffnet.\\}
    \lititem{\textbf{\sffamily Beschreibung:}\\}

    \singlespacing
\begin{enumerate}
    \item Die Anwendung wird geöffnet.
    \item \Gls{Login}-Bildschirm wird angezeigt.
    \item \Gls{Login} ist für die Nutzung der Anwendung erforderlich.
\end{enumerate}
\newpage
\end{beschreibung}
\onehalfspacing


\textbf{\large \sffamily{\Gls{Account} anlegen und einloggen}} $\langle$\textit F20$\rangle$

\begin{beschreibung}
    \lititem{\textbf{\sffamily Anwendungsfall:} Der Nutzer möchte sich anmelden.\\}
    \lititem{\textbf{\sffamily Anforderung:} MK1, MK12\\}
    \lititem{\textbf{\sffamily Ziel:} Der Nutzer legt einen neuen \Gls{Account} an und kann sich einloggen.\\}
    \lititem{\textbf{\sffamily Vorbedingung:} Der \Gls{Login}-Bildschirm wird angezeigt.\\}
    \lititem{\textbf{\sffamily Nachbedingung:} Erfolgreicher \Gls{Login}. Die Startseite wird angezeigt.\\}
    \lititem{\textbf{\sffamily Akteure:} Nutzer, Server\\}
    \lititem{\textbf{\sffamily Auslösendes Ereignis:} Öffnen der Anwendung, Drücken des \Gls{Login}-Buttons.\\}
    \lititem{\textbf{\sffamily Beschreibung:}\\}

    \singlespacing
\begin{enumerate}
    \item \Gls{Login}-Bildschirm wird angezeigt.
    \item Eingabe von Name, E-Mail und Passwort.
    \item Die Erstregistrierung muss von einem \Gls{Administrator} bestätigt werden.
\end{enumerate}
\vspace{1cm}
\end{beschreibung}
\onehalfspacing


\textbf{\large \sffamily{Schüler anzeigen}} $\langle$\textit F30$\rangle$

\begin{beschreibung}
    \lititem{\textbf{\sffamily Anwendungsfall:} Der Nutzer möchte die bereits angelegten Schüler verwalten.\\}
    \lititem{\textbf{\sffamily Anforderung:} MK2\\}
    \lititem{\textbf{\sffamily Ziel:} Anzeigen der Liste an Schülern.\\}
    \lititem{\textbf{\sffamily Vorbedingung:} Der Nutzer befindet sich auf der Startseite.\\}
    \lititem{\textbf{\sffamily Nachbedingung:} Die Schülerbibliothek wird geöffnet.\\}
    \lititem{\textbf{\sffamily Akteure:} Nutzer\\}
    \lititem{\textbf{\sffamily Auslösendes Ereignis:} Drücken des Buttons zum Öffnen der Schülerbibliothek.\\}
    \lititem{\textbf{\sffamily Beschreibung:}\\}

    \singlespacing
\begin{enumerate}
    \item Schülerbibliothek wird geöffnet.
    \item Liste der bereits angelegten Schüler wird angezeigt.
    \item Nutzer kann Schüler verwalten (löschen, hinzufügen und editieren).
    \item Durch Drücken auf einen Schüler werden weitere Informationen sowie Statistiken ($\langle$\textit F50$\rangle$) über ihn angezeigt.
\end{enumerate}
\newpage
\end{beschreibung}
\onehalfspacing


\textbf{\large \sffamily{Neuen Schüler anlegen}} $\langle$\textit F40$\rangle$

\begin{beschreibung}
    \lititem{\textbf{\sffamily Anwendungsfall:} Der Nutzer möchte einen neuen Schüler anlegen.\\}
    \lititem{\textbf{\sffamily Anforderung:} MK2\\}
    \lititem{\textbf{\sffamily Ziel:} Neuer Schüler wird erstellt und zur Schülerliste hinzugefügt.\\}
    \lititem{\textbf{\sffamily Vorbedingung:} Die Schülerbibliothek ist geöffnet.\\}
    \lititem{\textbf{\sffamily Nachbedingung:} Neu erstellter Schüler wird in der Schülerliste angezeigt.\\}
    \lititem{\textbf{\sffamily Akteure:} Nutzer\\}
    \lititem{\textbf{\sffamily Auslösendes Ereignis:} Drücken auf ''+''-Button in der Schülerliste.\\}
    \lititem{\textbf{\sffamily Beschreibung:}\\}

    \singlespacing
\begin{enumerate}
    \item Drücken auf ''+''-Button.
    \item Name des Schülers eingeben.
    \item (Optional) Der Nutzer kann den neu erstellten Nutzer direkt zu einem Kurs hinzufügen. ($\langle$\textit F90$\rangle$)
\end{enumerate}
\vspace{1cm}
\end{beschreibung}
\onehalfspacing


\textbf{\large \sffamily{Schülerstatistiken ansehen}} $\langle$\textit F50$\rangle$

\begin{beschreibung}
    \lititem{\textbf{\sffamily Anwendungsfall:} Der Nutzer möchte Statistiken über einen Schüler ansehen.\\}
    \lititem{\textbf{\sffamily Anforderung:} MK2, MK10, MK11, MK12\\}
    \lititem{\textbf{\sffamily Ziel:} Einsehen der Statistiken eines Schülers.\\}
    \lititem{\textbf{\sffamily Vorbedingung:} Die Schülerbibliothek ist geöffnet.\\}
    \lititem{\textbf{\sffamily Nachbedingung:} Die Statistiken über den ausgewählten Schüler werden angezeigt.\\}
    \lititem{\textbf{\sffamily Akteure:} Nutzer, Server\\}
    \lititem{\textbf{\sffamily Auslösendes Ereignis:} Drücken auf einen Schüler in der Schülerliste.\\}
    \lititem{\textbf{\sffamily Beschreibung:}\\}

    \singlespacing
\begin{enumerate}
    \item Den gewünschten Schüler in der Liste auswählen.
    \item Statistik über den Schüler wird geöffnet.
    \item Der Nutzer wird informiert, falls noch keine Statistiken zu dem Schüler verfügbar sind.
\end{enumerate}
\newpage
\end{beschreibung}
\onehalfspacing


\textbf{\large \sffamily{Kurse anzeigen}} $\langle$\textit F60$\rangle$

\begin{beschreibung}
    \lititem{\textbf{\sffamily Anwendungsfall:} Der Nutzer möchte die bereits angelegten Kurse verwalten.\\}
    \lititem{\textbf{\sffamily Anforderung:} MK3\\}
    \lititem{\textbf{\sffamily Ziel:} Anzeigen der Liste an Kursen.\\}
    \lititem{\textbf{\sffamily Vorbedingung:} Der Nutzer befindet sich auf der Startseite.\\}
    \lititem{\textbf{\sffamily Nachbedingung:} Die Kursbibliothek wird geöffnet.\\}
    \lititem{\textbf{\sffamily Akteure:} Nutzer\\}
    \lititem{\textbf{\sffamily Auslösendes Ereignis:} Drücken des Buttons zum Öffnen der Kursbibliothek.\\}
    \lititem{\textbf{\sffamily Beschreibung:}\\}

    \singlespacing
\begin{enumerate}
    \item Kursbibliothek wird geöffnet.
    \item Liste der bereits angelegten Kurse wird angezeigt.
    \item Nutzer kann Kurse verwalten (löschen, hinzufügen, editieren).
    \item Durch Drücken auf einen Kurs werden weitere Informationen sowie Statistiken ($\langle$\textit F80$\rangle$) über ihn angezeigt.
\end{enumerate}
\vspace{1cm}
\end{beschreibung}
\onehalfspacing


\textbf{\large \sffamily{Neuen Kurs anlegen}} $\langle$\textit F70$\rangle$

\begin{beschreibung}
    \lititem{\textbf{\sffamily Anwendungsfall:} Der Nutzer möchte einen neuen Kurs anlegen.\\}
    \lititem{\textbf{\sffamily Anforderung:} MK3\\}
    \lititem{\textbf{\sffamily Ziel:} Neuer Kurs wird erstellt und zur Kursliste hinzugefügt.\\}
    \lititem{\textbf{\sffamily Vorbedingung:} Die Kursbibliothek ist geöffnet.\\}
    \lititem{\textbf{\sffamily Nachbedingung:} Neu erstellter Kurs wird in der Kursliste angezeigt.\\}
    \lititem{\textbf{\sffamily Akteure:} Nutzer\\}
    \lititem{\textbf{\sffamily Auslösendes Ereignis:} Drücken auf ''+''-Button in der Kursliste.\\}
    \lititem{\textbf{\sffamily Beschreibung:}\\}

    \singlespacing
\begin{enumerate}
    \item Drücken auf ''+''-Button.
    \item Name des Kurses eingeben.
    \item (Optional) Der Nutzer kann ein Unterrichtsfach zum Kurs hinzufügen. ($\langle$\textit F220$\rangle$)
\end{enumerate}
\newpage
\end{beschreibung}
\onehalfspacing


\textbf{\large \sffamily{Kursstatistiken ansehen}} $\langle$\textit F80$\rangle$

\begin{beschreibung}
    \lititem{\textbf{\sffamily Anwendungsfall:} Der Nutzer möchte Statistiken über einen Kurs ansehen.\\}
    \lititem{\textbf{\sffamily Anforderung:} MK3, MK10, MK11, MK12\\}
    \lititem{\textbf{\sffamily Ziel:} Einsehen der Statistiken eines Kurses.\\}
    \lititem{\textbf{\sffamily Vorbedingung:} Die Kursbibliothek ist geöffnet.\\}
    \lititem{\textbf{\sffamily Nachbedingung:} Die Statistiken über den ausgewählten Kurs werden angezeigt.\\}
    \lititem{\textbf{\sffamily Akteure:} Nutzer, Server\\}
    \lititem{\textbf{\sffamily Auslösendes Ereignis:} Drücken auf einen Kurs in der Kursliste.\\}
    \lititem{\textbf{\sffamily Beschreibung:}\\}

    \singlespacing
\begin{enumerate}
    \item Den gewünschten Kurs in der Kursliste auswählen.
    \item Statistik über den Kurs wird geöffnet.
    \item Der Nutzer wird informiert, falls noch keine Statistiken zu dem Kurs verfügbar sind. 
\end{enumerate}
\vspace{1cm}
\end{beschreibung}
\onehalfspacing


\textbf{\large \sffamily{Schüler in einem Kurs anzeigen}} $\langle$\textit F90$\rangle$

\begin{beschreibung}
    \lititem{\textbf{\sffamily Anwendungsfall:} Der Nutzer möchte die Schüler in einem Kurs anzeigen.\\}
    \lititem{\textbf{\sffamily Anforderung:} MK3, MK4\\}
    \lititem{\textbf{\sffamily Ziel:} Anzeigen der Liste an Schülern in einem Kurs.\\}
    \lititem{\textbf{\sffamily Vorbedingung:} Ein Kurs wurde in der Kursliste ausgewählt.\\}
    \lititem{\textbf{\sffamily Nachbedingung:} Schülerliste des Kurses wird geöffnet.\\}
    \lititem{\textbf{\sffamily Akteure:} Nutzer\\}
    \lititem{\textbf{\sffamily Auslösendes Ereignis:} Drücken auf Schülerliste im Kurs.\\}
    \lititem{\textbf{\sffamily Beschreibung:}\\}

    \singlespacing
\begin{enumerate}
    \item Die Schülerliste des Kurses wird geöffnet.
    \item Nutzer kann die Schüler des Kurses ansehen und verwalten (hinzufügen, löschen).
    \item Durch Drücken auf einen Schüler werden weitere Informationen sowie Statistiken ($\langle$\textit F50$\rangle$) über ihn angezeigt.
\end{enumerate}
\newpage
\end{beschreibung}
\onehalfspacing


\textbf{\large \sffamily{Schüler zu einem Kurs hinzufügen}} $\langle$\textit F100$\rangle$

\begin{beschreibung}
    \lititem{\textbf{\sffamily Anwendungsfall:} Der Nutzer möchte einen Schüler zu einem Kurs hinzufügen. \\}
    \lititem{\textbf{\sffamily Anforderung:} MK3, MK4\\}
    \lititem{\textbf{\sffamily Ziel:} Ein Schüler wird zu einem Kurs hinzugefügt.\\}
    \lititem{\textbf{\sffamily Vorbedingung:} Die Schülerliste des Kurses ist geöffnet.\\}
    \lititem{\textbf{\sffamily Nachbedingung:} Der hinzugefügte Schüler wird in der Liste angezeigt.\\}
    \lititem{\textbf{\sffamily Akteure:} Nutzer\\}
    \lititem{\textbf{\sffamily Auslösendes Ereignis:} Drücken auf ''+''-Button in der Schülerliste des Kurses.\\}
    \lititem{\textbf{\sffamily Beschreibung:}\\}

    \singlespacing
\begin{enumerate}
    \item Drücken auf ''+''-Button.
    \item Schüler aus Schülerbibliothek auswählen.
    \item Ein Schüler kann zu mehreren Kursen hinzugefügt werden.
\end{enumerate}
\vspace{1cm}
\end{beschreibung}
\onehalfspacing


\textbf{\large \sffamily{Schüler aus Kurs entfernen}} $\langle$\textit F110$\rangle$

\begin{beschreibung}
    \lititem{\textbf{\sffamily Anwendungsfall:} Der Nutzer möchte einen Schüler aus einem Kurs entfernen.\\}
    \lititem{\textbf{\sffamily Anforderung:} MK3, MK4\\}
    \lititem{\textbf{\sffamily Ziel:} Ein Schüler wird aus dem Kurs entfernt.\\}
    \lititem{\textbf{\sffamily Vorbedingung:} Die Schülerliste des Kurses ist geöffnet.\\}
    \lititem{\textbf{\sffamily Nachbedingung:} \\}
    \lititem{\textbf{\sffamily Akteure:} Nutzer\\}
    \lititem{\textbf{\sffamily Auslösendes Ereignis:} Drücken auf Entfernen-Button neben dem Schüler in der Schülerliste des Kurses.\\}
    \lititem{\textbf{\sffamily Beschreibung:}\\}

    \singlespacing
\begin{enumerate}
    \item Der Schüler wird aus dem Kurs entfernt.
    \item Der Schüler existiert weiterhin in der Schülerbibliothek.
\end{enumerate}
\newpage
\end{beschreibung}
\onehalfspacing


\textbf{\large \sffamily{Neue Sitzung erstellen}} $\langle$\textit F120$\rangle$

\begin{beschreibung}
    \lititem{\textbf{\sffamily Anwendungsfall:} Der Nutzer möchte eine neue Sitzung für die Interaktionsaufzeichnung erstellen.\\}
    \lititem{\textbf{\sffamily Anforderung:} MK5, MK5\\}
    \lititem{\textbf{\sffamily Ziel:} Der Nutzer erstellt eine neue Interaktionskarte für den gewählten Kurs.\\}
    \lititem{\textbf{\sffamily Vorbedingung:} Der Schüler befindet sich auf der Startseite.\\}
    \lititem{\textbf{\sffamily Nachbedingung:} Eine neue Sitzung wird erstellt. Die Schüler des Kurses werden angezeigt und der Nutzer kann Interaktionen aufzeichnen.\\}
    \lititem{\textbf{\sffamily Akteure:} Nutzer\\}
    \lititem{\textbf{\sffamily Auslösendes Ereignis:} ''Neue Sitzung''-Button gedrückt.\\}
    \lititem{\textbf{\sffamily Beschreibung:}\\}

    \singlespacing
\begin{enumerate}
    \item Den Kurs für die Interaktionsaufzeichnung wählen.
    \item Eine neue Sitzung wird erstellt.
    \item Interaktionen können aufgezeichnet werden. ($\langle$\textit F130$\rangle$)
\end{enumerate}
\vspace{1cm}
\end{beschreibung}
\onehalfspacing


\textbf{\large \sffamily{Interaktion aufzeichnen}} $\langle$\textit F130$\rangle$

\begin{beschreibung}
    \lititem{\textbf{\sffamily Anwendungsfall:} Der Nutzer möchte eine Interaktion von Schülern aufzeichnen.\\}
    \lititem{\textbf{\sffamily Anforderung:} MK5, MK6, MK7, MK8\\}
    \lititem{\textbf{\sffamily Ziel:} Interaktionen zwischen Schülern sowie zwischen Schülern und Lehrer erfassen.\\}
    \lititem{\textbf{\sffamily Vorbedingung:} Eine neue Sitzung wurde erstellt.\\}
    \lititem{\textbf{\sffamily Nachbedingung:} Ein Fenster für die Kategorieauswahl öffnet sich.\\}
    \lititem{\textbf{\sffamily Akteure:} Nutzer\\}
    \lititem{\textbf{\sffamily Auslösendes Ereignis:} Auf einen Schüler drücken.\\}
    \lititem{\textbf{\sffamily Beschreibung:}\\}

    \singlespacing
\begin{enumerate}
    \item Interaktion zwischen Schülern/Lehrer wird erfasst.
    \item Die Interaktion kann einer Kategorie zugeordnet werden. ($\langle$\textit F140$\rangle$)
\end{enumerate}
\newpage
\end{beschreibung}
\onehalfspacing


\textbf{\large \sffamily{Kategorie für Interaktion auswählen}} $\langle$\textit F140$\rangle$

\begin{beschreibung}
    \lititem{\textbf{\sffamily Anwendungsfall:} Der Nutzer möchte Interaktionen von Schülern bewerten und einer Kategorie zuordnen.\\}
    \lititem{\textbf{\sffamily Anforderung:} MK5, MK9\\}
    \lititem{\textbf{\sffamily Ziel:} Die Interaktion wird einer Kategorie zugeordnet.\\}
    \lititem{\textbf{\sffamily Vorbedingung:} Eine Interaktion wurde erfasst und das Fenster für die Kategorieauswahl ist geöffnet.\\}
    \lititem{\textbf{\sffamily Nachbedingung:} Die Kategorie wurde ausgewählt. Die Schüler in der Sitzung werden wieder angezeigt und es können weitere Interaktionen aufgezeichnet werden.\\}
    \lititem{\textbf{\sffamily Akteure:} Nutzer\\}
    \lititem{\textbf{\sffamily Auslösendes Ereignis:} Interaktion zwischen Schülern wurde erfasst.\\}
    \lititem{\textbf{\sffamily Beschreibung:}\\}

    \singlespacing
\begin{enumerate}
    \item Die Interaktion wird einer Kategorie zugeordnet.
    \item Der Nutzer kann ggf. eigene Kategorien definieren. ($\langle$\textit F250$\rangle$)
\end{enumerate}
\vspace{1cm}
\end{beschreibung}
\onehalfspacing


\textbf{\large \sffamily{Interaktionskarten speichern}} $\langle$\textit F150$\rangle$

\begin{beschreibung}
    \lititem{\textbf{\sffamily Anwendungsfall:} Der Nutzer möchte die Interaktionskarte einer Sitzung speichern.\\}
    \lititem{\textbf{\sffamily Anforderung:} MK5, MK12\\}
    \lititem{\textbf{\sffamily Ziel:} Die Interaktionskarte wird in den Aufzeichnungen gespeichert und kann jederzeit angesehen werden.\\}
    \lititem{\textbf{\sffamily Vorbedingung:} Eine Interaktionskarte wurde in einer Sitzung erstellt.\\}
    \lititem{\textbf{\sffamily Nachbedingung:} Die Interaktionskarte ist in den Aufzeichnungen einsehbar.\\}
    \lititem{\textbf{\sffamily Akteure:} Nutzer\\}
    \lititem{\textbf{\sffamily Auslösendes Ereignis:} Eine Sitzung wird beendet.\\}
    \lititem{\textbf{\sffamily Beschreibung:}\\}

    \singlespacing
\begin{enumerate}
    \item Die Interaktionskarte wird in den Aufzeichnungen gespeichert.
    \item Interaktionskarten werden bei vorhandener Internetverbindung zwischen Geräten synchronisiert.
    \item (Optional) Der Nutzer kann die erstellte Interaktionskarte auf Wunsch direkt exportieren. ($\langle$\textit F180$\rangle$)
\end{enumerate}
\newpage
\end{beschreibung}
\onehalfspacing


\textbf{\large \sffamily{Interaktionskarten ansehen}} $\langle$\textit F160$\rangle$

\begin{beschreibung}
    \lititem{\textbf{\sffamily Anwendungsfall:} Der Nutzer möchte Interaktionskarten von vorherigen Sitzungen ansehen.\\}
    \lititem{\textbf{\sffamily Anforderung:} MK5, MK6, MK7, MK12\\}
    \lititem{\textbf{\sffamily Ziel:} Interaktionen von vergangenen Sitzungen einsehen.\\}
    \lititem{\textbf{\sffamily Vorbedingung:} Der Nutzer befindet sich auf der Startseite.\\}
    \lititem{\textbf{\sffamily Nachbedingung:} Eine Übersicht über die gespeicherten Interaktionskarten wird geöffnet.\\}
    \lititem{\textbf{\sffamily Akteure:} Nutzer, Server\\}
    \lititem{\textbf{\sffamily Auslösendes Ereignis:} Der Menüpunkt ''Interaktionskarten'' wird gedrückt.\\}
    \lititem{\textbf{\sffamily Beschreibung:}\\}

    \singlespacing
\begin{enumerate}
    \item Übersicht der Interaktionskarten wird geöffnet.
    \item Nutzer kann Interaktionskarten ansehen und löschen.
    \item Durch Drücken auf eine Interaktionskarte werden weitere Informationen sowie die Statistiken der Sitzung ($\langle$\textit F170$\rangle$) angezeigt.
\end{enumerate}
\vspace{1cm}
\end{beschreibung}
\onehalfspacing


\textbf{\large \sffamily{Sitzungsstatistiken ansehen}} $\langle$\textit F170$\rangle$

\begin{beschreibung}
    \lititem{\textbf{\sffamily Anwendungsfall:} Der Nutzer möchte Statistiken zu vergangenen Sitzungen einsehen.\\}
    \lititem{\textbf{\sffamily Anforderung:} MK5, MK10, MK12\\}
    \lititem{\textbf{\sffamily Ziel:} Anzeigen der Statistiken einer vergangenen Sitzung.\\}
    \lititem{\textbf{\sffamily Vorbedingung:} Die Übersicht über die Interaktionskarten ist geöffnet.\\}
    \lititem{\textbf{\sffamily Nachbedingung:} Die Statistiken über die ausgewählte Sitzung werden angezeigt.\\}
    \lititem{\textbf{\sffamily Akteure:} Nutzer, Server\\}
    \lititem{\textbf{\sffamily Auslösendes Ereignis:} Drücken auf eine Interaktionskarte in der Übersicht.\\}
    \lititem{\textbf{\sffamily Beschreibung:}\\}

    \singlespacing
\begin{enumerate}
    \item Interaktionskarte der gewünschten Sitzung auswählen.
    \item Statistik über die Sitzung wird geöffnet.
\end{enumerate}
\newpage
\end{beschreibung}
\onehalfspacing


\textbf{\large \sffamily{Interaktionskarten exportieren}} $\langle$\textit F180$\rangle$

\begin{beschreibung}
    \lititem{\textbf{\sffamily Anwendungsfall:} Der Nutzer möchte Interaktionskarten exportieren, um sie auch ohne Internetverbindung einsehen zu können.\\}
    \lititem{\textbf{\sffamily Anforderung:} MK5, MK13\\}
    \lititem{\textbf{\sffamily Ziel:} Interaktionskarten in einem geeigneten Format exportieren.\\}
    \lititem{\textbf{\sffamily Vorbedingung:} Eine Interaktionskarte wurde in der Übersicht ausgewählt.\\}
    \lititem{\textbf{\sffamily Nachbedingung:} Die Interaktionskarte wurde erfolgreich im festgelegten Format exportiert.\\}
    \lititem{\textbf{\sffamily Akteure:} Nutzer\\}
    \lititem{\textbf{\sffamily Auslösendes Ereignis:} ''Exportieren''-Button wird gedrückt.\\}
    \lititem{\textbf{\sffamily Beschreibung:}\\}

    \singlespacing
\begin{enumerate}
    \item Die ausgewählte Interaktionskarte wird im festgelegten Format exportiert.
    \item Interaktionskarten können auch direkt nach Beenden einer Sitzung exportiert werden.
\end{enumerate}
\vspace{1cm}
\end{beschreibung}
\onehalfspacing


\textbf{\large \sffamily{Schüler teilen}} $\langle$\textit F190$\rangle$

\begin{beschreibung}
    \lititem{\textbf{\sffamily Anwendungsfall:} Der Nutzer möchte Schüler mit anderen \Gls{Account}s (Lehrern) teilen.\\}
    \lititem{\textbf{\sffamily Anforderung:} KK1, MK1, MK2\\}
    \lititem{\textbf{\sffamily Ziel:} Ein Schüler wird mit einem anderen \Gls{Account} geteilt und dort automatisch in der Schülerbibliothek angelegt. Alle Informationen und Statistiken des Schülers werden übernommen.\\}
    \lititem{\textbf{\sffamily Vorbedingung:} Der gewünschte Schüler wurde in der Schülerbibliothek ausgewählt. Die E-Mail des \Gls{Account}s, mit dem der Schüler geteilt werden soll, ist bekannt.\\}
    \lititem{\textbf{\sffamily Nachbedingung:} Der geteilte Schüler wird bei dem anderen \Gls{Account} in der Schülerbibliothek angezeigt\\}
    \lititem{\textbf{\sffamily Akteure:} Nutzer, Server\\}
    \lititem{\textbf{\sffamily Auslösendes Ereignis:} Drücken auf ''Teilen''-Button in Schüleransicht.\\}
    \lititem{\textbf{\sffamily Beschreibung:}\\}

    \singlespacing
\begin{enumerate}
    \item E-Mail des \Gls{Account}s angeben, mit dem der Schüler geteilt werden soll.
    \item Der Schüler wird automatisch angelegt.
\end{enumerate}
\newpage
\end{beschreibung}
\onehalfspacing

\textbf{\large \sffamily{Kurse teilen}} $\langle$\textit F200$\rangle$

\begin{beschreibung}
    \lititem{\textbf{\sffamily Anwendungsfall:} Der Nutzer möchte Kurse mit anderen \Gls{Account}s (Lehrern) teilen.\\}
    \lititem{\textbf{\sffamily Anforderung:} KK1, MK1, MK3\\}
    \lititem{\textbf{\sffamily Ziel:} Ein Kurs wird mit einem anderen \Gls{Account} geteilt und dort automatisch in der Kursbibliothek angelegt. Alle Informationen und Statistiken des Kurses werden übernommen.\\}
    \lititem{\textbf{\sffamily Vorbedingung:} Der gewünschte Kurs wurde in der Kursbibliothek ausgewählt. Die E-Mail des \Gls{Account}s, mit dem der Kurs geteilt werden soll, ist bekannt.\\}
    \lititem{\textbf{\sffamily Nachbedingung:} Der geteilte Kurs wird bei dem anderen \Gls{Account} in der Kursbibliothek angezeigt.\\}
    \lititem{\textbf{\sffamily Akteure:} Nutzer, Server\\}
    \lititem{\textbf{\sffamily Auslösendes Ereignis:} Drücken auf ''Teilen''-Button in der Kursansicht.\\}
    \lititem{\textbf{\sffamily Beschreibung:}\\}

    \singlespacing
\begin{enumerate}
    \item E-Mail des \Gls{Account}s angeben, mit dem der Kurs geteilt werden soll.
    \item Der Kurs wird automatisch angelegt. Befinden sich in dem Kurs Schüler, die noch nicht in der Schülerbibliothek existieren, werden diese ebenfalls angelegt.
\end{enumerate}
\vspace{1cm}
\end{beschreibung}
\onehalfspacing


\textbf{\large \sffamily{Sitzordnung für einen Kurs anlegen}} $\langle$\textit F210$\rangle$

\begin{beschreibung}
    \lititem{\textbf{\sffamily Anwendungsfall:} Der Nutzer möchte eine Sitzordnung für einen Kurs anlegen.\\}
    \lititem{\textbf{\sffamily Anforderung:} KK2, KK3, MK3\\}
    \lititem{\textbf{\sffamily Ziel:} Anlegen einer festen Sitzordnung der Schüler innerhalb eines Kurses.\\}
    \lititem{\textbf{\sffamily Vorbedingung:} Der gewünschte Kurs wurde in der Kursbibliothek ausgewählt. Der Nutzer befindet sich in der Desktop-Ansicht der Anwendung.\\}
    \lititem{\textbf{\sffamily Nachbedingung:} Die Sitzordnung des Kurses wird in der Desktop-Ansicht von Sitzungen angezeigt.\\}
    \lititem{\textbf{\sffamily Akteure:} Nutzer\\}
    \lititem{\textbf{\sffamily Auslösendes Ereignis:} Drücken auf ''Sitzordnung erstellen''-Button in der Kursansicht.\\}
    \lititem{\textbf{\sffamily Beschreibung:}\\}

    \singlespacing
\begin{enumerate}
    \item Auswählen eines Kurses in der Kursbibliothek.
    \item Drücken auf ''Sitzordnung erstellen''-Button.
\end{enumerate}
\newpage
\end{beschreibung}
\onehalfspacing


\textbf{\large \sffamily{Unterrichtsfach zu einem Kurs hinzufügen}} $\langle$\textit F220$\rangle$

\begin{beschreibung} 
    \lititem{\textbf{\sffamily Anwendungsfall:} Der Nutzer möchte ein Unterrichtsfach zu einem Kurs hinzufügen.\\}
    \lititem{\textbf{\sffamily Anforderung:} KK4, MK3\\}
    \lititem{\textbf{\sffamily Ziel:} Dem Kurs wird ein Unterrichtsfach zugeordnet.\\}
    \lititem{\textbf{\sffamily Vorbedingung:} Der gewünschte Kurs wurde in der Kursbibliothek ausgewählt.\\}
    \lititem{\textbf{\sffamily Nachbedingung:} Das Unterrichtsfach wird in der Kursansicht angezeigt.\\}
    \lititem{\textbf{\sffamily Akteure:} Nutzer\\}
    \lititem{\textbf{\sffamily Auslösendes Ereignis:} Drücken auf ''Fach hinzufügen''-Button in der Kursansicht.\\}
    \lititem{\textbf{\sffamily Beschreibung:}\\}

    \singlespacing
\begin{enumerate}
    \item Auswählen eines Kurses in der Kursbibliothek.
    \item Drücken auf ''Fach hinzufügen''-Button.
    \item Kurse können in der Kursbibliothek nach dem Fach sortiert werden.
\end{enumerate}
\vspace{1cm}
\end{beschreibung}
\onehalfspacing

\textbf{\large \sffamily{Interaktion rückgängig machen mit undo}} $\langle$\textit F230$\rangle$

\begin{beschreibung}
    \lititem{\textbf{\sffamily Anwendungsfall:} Der Nutzer möchte eine eingegebene Interaktion zwischen Schülern rückgängig machen. Dies ist beispielsweise der Fall, wenn der Nutzer eine falsche Eingabe getätigt hat.\\}
    \lititem{\textbf{\sffamily Anforderung:} KK5, MK5\\}
    \lititem{\textbf{\sffamily Ziel:} Löschen einer eingegebenen Interaktion aus der Interaktionskarte.\\}
    \lititem{\textbf{\sffamily Vorbedingung:} Eine Interaktion wurde aufgezeichnet.\\}
    \lititem{\textbf{\sffamily Nachbedingung:} Die Interaktion wird aus der Interaktionskarte gelöscht.\\}
    \lititem{\textbf{\sffamily Akteure:} Nutzer\\}
    \lititem{\textbf{\sffamily Auslösendes Ereignis:} Drücken auf ''\Gls{undo}''-Button.\\}
    \lititem{\textbf{\sffamily Beschreibung:}\\}
    
    \singlespacing
\begin{enumerate}
    \item Die zuletzt eingegebene Interaktion wird rückgängig gemacht.
    \item Die Interaktion wird aus der Interaktionskarte gelöscht und wird in der Statistik der Sitzung nicht beachtet.
\end{enumerate}
\newpage
\end{beschreibung}
\onehalfspacing


\textbf{\large \sffamily{Qualität bei Interaktion angeben}} $\langle$\textit F240$\rangle$

\begin{beschreibung}
    \lititem{\textbf{\sffamily Anwendungsfall:} Der Nutzer möchte Interaktionen von Schülern in Hinsicht auf Qualität bewerten (zum Beispiel die Komplexität einer Antwort).\\}
    \lititem{\textbf{\sffamily Anforderung:} KK4, MK5\\}
    \lititem{\textbf{\sffamily Ziel:} Bewertung der Interaktion eines Schülers.\\}
    \lititem{\textbf{\sffamily Vorbedingung:} Eine Interaktion wurde erfasst und das Fenster für die Qualitätsangabe ist geöffnet.\\}
    \lititem{\textbf{\sffamily Nachbedingung:} Die Qualität wurde angegeben. Die Schüler in der Sitzung werden wieder angezeigt und es können weitere Interaktionen aufgezeichnet werden\\}
    \lititem{\textbf{\sffamily Akteure:} Nutzer\\}
    \lititem{\textbf{\sffamily Auslösendes Ereignis:} Eine Interaktion wurde erfasst.\\}
    \lititem{\textbf{\sffamily Beschreibung:}\\}

    \singlespacing
\begin{enumerate}
    \item Nach Erfassen einer Interaktion kann die Qualität angegeben werden.
    \item Die Qualität wird mit einem Sternesystem bewertet.
\end{enumerate}
\vspace{1cm}
\end{beschreibung}
\onehalfspacing


\textbf{\large \sffamily{Eigene Kategorie für Interaktionen erstellen}} $\langle$\textit F250$\rangle$

\begin{beschreibung}
    \lititem{\textbf{\sffamily Anwendungsfall:} Der Nutzer möchte eigene Kategorien für Interaktionen erstellen.\\}
    \lititem{\textbf{\sffamily Anforderung:} KK7, MK5, MK9\\}
    \lititem{\textbf{\sffamily Ziel:} Eine benutzerdefinierte Interaktionskategorie wird erstellt.\\}
    \lititem{\textbf{\sffamily Vorbedingung:} Eine Interaktion wurde erfasst und das Fenster für die Kategorieauswahl ist geöffnet.\\}
    \lititem{\textbf{\sffamily Nachbedingung:} Die neu erstellte Kategorie wurde zur Kategorieauswahl hinzugefügt.\\}
    \lititem{\textbf{\sffamily Akteure:} Nutzer\\}
    \lititem{\textbf{\sffamily Auslösendes Ereignis:} Drücken auf ''+''-Button bei Kategorieauswahl.\\}
    \lititem{\textbf{\sffamily Beschreibung:}\\}

    \singlespacing
\begin{enumerate}
    \item Nach Erfassen einer Interaktion kann bei der Auswahl einer Kategorie eine eigene erstellt werden.
    \item Die neu erstellte Kategorie wird gespeichert und ist von nun an bei jeder Interaktion verfügbar.
\end{enumerate}
\end{beschreibung}
\onehalfspacing