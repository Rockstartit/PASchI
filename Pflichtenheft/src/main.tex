\documentclass[parskip=full]{scrartcl}
\usepackage[utf8]{inputenc} % use utf8 file encoding for TeX sources
\usepackage{graphicx}
\usepackage[T1]{fontenc} % avoid garbled Unicode text in pdf
\usepackage[german]{babel} % german hyphenation, quotes, etc
\usepackage{hyperref} % detailed hyperlink/pdf configuration
\usepackage{csquotes} % provides \enquote{} macro for "quotes"
\usepackage{float}
\usepackage{svg}
\usepackage{enumitem} % Aufzählungen mit Zahlen
\usepackage{setspace}
\usepackage[toc,section=section]{glossaries}

\usepackage{fancyhdr} % Paket zum Bearbeiten der Fußzeile

\counterwithin{figure}{section}

\hypersetup{                % ‘texdoc hyperref‘ for options
  pdftitle={Pflichtenheft},
  pdfauthor={Florian Knechtel, Luka Kosak, David Maier, Cián Payne, Aaron Sutor}
  }


\title{PAschI: Programm zur Aufzeichnung schülerischer Interaktionen}
\author{Florian Knechtel, Luka Kosak, David Maier, Cián Payne, Aaron Sutor}
\date{Dezember 2022}

\makeatletter %Command für \@befehle

\fancyhf{} % Alle Kopf- und Fußzeilenfelder bereinigen
\pagestyle{fancy} % Eigener Seitenstil

\fancyfoot[C]{\thepage} 
\fancyhead[L]{\large\textsc{PAschI: Interaktions-App für Classroom-Management}}
\renewcommand{\headrulewidth}{0.7pt}

\setkomafont{section}{\LARGE}
\setkomafont{subsection}{\Large}
\renewcommand{\sfdefault}{qhv}


\newcommand{\glossaryentry}[2] {
    \newglossaryentry{#1}{
        name={#1},
        description={#2}
    }
}

\glossaryentry{PWA}{Progressive Web App. Technologie, die es ermöglicht, Webapplikationen auf Geräten mit unterstützendem \Gls{Browser} installieren zu können, sodass diese auch offline funktionieren (mit einem eigenen \Gls{Cache}, der mithilfe von Service-Workern mit dem Server synchronisiert werden kann)}
\glossaryentry{Frontend}{Das Frontend ist in einer Schichteneinteilung der Teil des Programms, der näher am Benutzer ist}
\glossaryentry{Backend}{Das Backend ist in einer Schichteneinteilung der Teil des Programms, der näher am System ist}
\glossaryentry{Browser}{Anwendung, die die Darstellung von Seiten im World Wide Web ermöglicht}
\glossaryentry{Server}{Der Begriff Server beschreibt einen großen und leistungsstarken Computer, auf dem ein Programm läuft, dass anderen Programmen im Rahmen eines Client-Server Modells Dienste zentral zur Verfügung stellt}
\glossaryentry{TypeScript}{TypeScript ist eine Programmiersprache zur Implementierung von Logik auf Internetseiten}
\glossaryentry{HTML5}{Hypertext Markup Language 5 ist die fünfte Version einer Computersprache zur Auszeichnung und Gestaltung von Texten und anderen Inhalten. Sie enthält keine Logik}
\glossaryentry{Framework}{Ein Framework ist ein Rahmen für eine Anwendung. Es ist selbst noch kein vollständiges Programm, sondern vielmehr ein Werkzeug für den Programmierer, um ein Programm einfacher und schneller entwickeln zu können}
\glossaryentry{VM}{Eine virtuelle Maschine ist ein Softwareprogramm, die einen Computer simuliert}
\glossaryentry{Java}{Java ist eine objektorientierte Programmiersprache, die geräteunabhängig ausgeführt werden kann}
\glossaryentry{GUI}{Die Grafische Benutzeroberfläche bezeichnet eine \Gls{Schnittstelle} zwischen dem Benutzer und dem jeweiligen Gerät}
\glossaryentry{Account}{Ein Benutzerkonto (engl. Account) ist eine Zugangsberechtigung zu einem System}
\glossaryentry{Dashboard}{Das Dashboard ist eine grafische Oberfläche zur übersichtlichen Visualisierung von Daten}
\glossaryentry{undo}{Die undo-Funktion bezeichnet eine Funktion in einem Programm, mit der die letzte ausgeführte Aktion rückgängig gemacht werden kann}
\glossaryentry{Themes}{Ein Theme ist eine vorgefertigte Zusammenstellung grafischer Elemente, bspw. von Farben, Schriftart oder Schriftgröße}
\glossaryentry{System Usability Scale}{Der System Usability Scale basiert auf einem einfachen Fragebogen, der technologieunabhängig das Maß der Gebrauchsfertigkeit eines Systems bewertet. Durch Auswerten der Fragebögen wird ein Score bestimmt, der zwischen null und 100 liegen kann. Je höher der Score, desto besser ist das System zu benutzen. 68 Punkte entsprechen einem guten Ergebnis}
\glossaryentry{Live}{Live bedeutet in Echtzeit}
\glossaryentry{Desktop}{Mit Desktop ist im Allgemeinen ein Computer gemeint. Hier spielt vor allem die meist größere Ansicht eine Rolle}
\glossaryentry{Chromium}{Chromium ist ein \Gls{Browser}, der als Grundlage vieler anderer Browser gilt. Es ist von dem Unternehmen Google}
\glossaryentry{Synchronisation}{Als Synchronisation bezeichnet man den Abgleich von Daten via Internetverbindung}
\glossaryentry{Android}{Ein Betriebssystem für Smartphones. Siehe \href{https://source.android.com/ }{Android Open Source Project}}
\glossaryentry{iOS}{Ein Betriebssystem für Smartphones des Unternehmens Apple}
\glossaryentry{Display}{Das Display ist eine Technik zum Anzeigen von digitalen Informationen}
\glossaryentry{Vue.js}{Spring-Boot ist ein \Gls{Frontend}-\Gls{Framework} für die Programmiersprachen \Gls{TypeScript} und \Gls{Javascript}}
\glossaryentry{Spring-Boot}{Spring-Boot ist ein \Gls{Backend}-\Gls{Framework} für die Programmiersprachen \Gls{Java} und \Gls{Kotlin}}
\glossaryentry{Datenbank}{Eine Datenbank ist ein System, welches Daten speichern, verwalten und wiedergeben kann}
\glossaryentry{Rest}{Representational State Transfer ist ein Architekturstil für die Erstellung von \Gls{API}s. Rest dient zur Maschine-zur-Maschine-Kommunikation}
\glossaryentry{API}{Eine API ist eine \Gls{Schnittstelle}, die von einem Softwaresystem zur Verfügung gestellt wird und anderen Programmen eine Anbindung an das System zugeben}
\newglossaryentry{Schnittstelle}{
name={Schnittstelle},
description={Die Schnittstelle ist ein Programmteil, das zur Kommunikation dient},
text={schnittstelle}}
\glossaryentry{Kotlin}{Kotlin ist eine Programmiersprache mit häufigem Einsatz in der Entwicklung von \Gls{Android}- und \gls{iOS}-Apps. Sie läuft mit der Java Virtuell Machine}
\glossaryentry{Javascript}{Javascript ist eine Programmiersprache, um hauptsächlich Logik auf Internetseiten zu implementieren}
\newglossaryentry{gehasht}{
name=Hashing,
description=Passwörter werden mittels eines Passwort-Hashing-Verfahrens in eine festgelegte Codefolge mit zufälligen Zahlen und Buchstaben umgewandelt,
text=gehasht}
\glossaryentry{SSL}{Secure Sockets Layer (SSL) ist ein Protokoll zur Verschlüsselung der Kommunikation im Internet}
\glossaryentry{HTTP}{Hypertext Transfer Protocol (HTTP) ist ein Protokoll zur Übertragung von Daten in einem Rechnernetz. Es wird insbesondere verwendet, um Webseiten aus dem World Wide Web in den \Gls{Browser} zu laden}
\glossaryentry{MySQL}{MySQL ist ein Open-Source Datenbankverwaltungssystem}
\glossaryentry{Open-Source}{Als Open-Source bezeichnet man Programme/Programmquelltext der von Dritten eingelesen, genutzt und geändert werden}
\glossaryentry{Swipe}{Das Wischen (engl. swipe) bezeichnet die "Wisch"-Bewegung bei der Bedienung eines Touchscreens}
\glossaryentry{Touchscreen}{Der Touchscreen ist ein \Gls{Display}, welches ein zusätzliches Eingabegerät für Berührungen hat}
\glossaryentry{Login}{Der Login bezeichnet das Anmelden an einem digitalen System}
\glossaryentry{Administrator}{Verwalter des Produktes mit erweiterten Funktionen}
\glossaryentry{Safari}{Safari ist ein \Gls{Browser}, der auf dem Betriebssystem \gls{iOS} vorinstalliert ist}
\glossaryentry{Cache}{Ein Cache ist ein schneller Pufferspeicher eines Computers}

\makenoidxglossaries

\begin{document}

\begin{titlepage}
    \begin{center}
        \vspace*{1cm}
        \Huge
        \textbf{\textsc{\@title}}
        \vspace{0.5cm}
        \LARGE
        \vspace{1.5cm}
        
        \textbf{\@author}
        \vspace{2cm}

        \Huge
        \textbf{Pflichtenheft}
        \vfill
        \includesvg[inkscapelatex=false,width=0.4\textwidth]{graphics/Logo_KIT.svg}
        
        \Large
        Institut für Informationssicherheit und Verlässlichkeit\\
        Karlsruher Institut für Technologie\\
        \@date
            
    \end{center}
    \setlength{\textheight}{23.5cm}
\end{titlepage}

\setlength{\textheight}{23.5cm}
\tableofcontents
\onehalfspacing

\section{Einleitung}
    Im Klassenzimmer den Überblick zu behalten, ist für Lehrkräfte eine große Herausforderung. Vor allem,
    wenn es dann um Einschätzungen von Leistungspotential und Mitarbeit einzelner Schüler:innen (z.B. für
    mündliche Noten) sowie von Rahmenbedingungen wie Gruppen- und Klassendynamik (z.B. Störfaktoren)
    geht. Hier müssen sich die meisten dann auf ihre subjektive Wahrnehmung und ihr Erinnerungsvermögen
    verlassen. Dabei spielen gerade solche Aspekte eine wichtige Rolle dabei, eine Lernatmosphäre zu schaffen,
    von der alle Lernenden optimal profitieren. Hierzu braucht es neue Möglichkeiten zur Erfassung des Unterrichtsgeschehens, die eine objektivere Einschätzung der Unterrichts- und Lernsituation auch im Nachhinein
    ermöglichen. Solche Tools müssen aber gleichzeitig unkompliziert und einfach in der Handhabung sein.
    In diesem Projekt soll eine Interaktions-App entwickelt werden, die das Classroom-Management für Lehrkräfte im oben umrissenen Nutzfall vereinfachen soll.

\section{Zielbestimmung}

\subsection{Musskriterien}
\label{list:Musskriterien}

\begin{tabular}{p{1.1cm}p{12.9cm}}
     $\langle$\textit{MK}1$\rangle$ & Der Benutzer muss einen \Gls{Account} anlegen und sich anmelden können.\\
     $\langle$\textit{MK}2$\rangle$ & Der Benutzer muss Schülerprofile anlegen können.\\
     $\langle$\textit{MK}3$\rangle$ & Schüler müssen zu Kursen zusammengefasst werden können.\\
     $\langle$\textit{MK}4$\rangle$ & Man muss Schüler zu Kursen hinzufügen und sie aus Kursen entfernen können.\\
     $\langle$\textit{MK}5$\rangle$ & Interaktionen zwischen Schülern sowie zwischen Schülern und Lehrern müssen grafisch erfasst werden können.\\
     $\langle$\textit{MK}6$\rangle$ & Interaktionen müssen visuell sichtbar sein.\\
     $\langle$\textit{MK}7$\rangle$ & Man muss die Richtung der Interaktion angeben können.\\
     $\langle$\textit{MK}8$\rangle$ & Interaktionen müssen über einen Zeitstempel verfügen.\\
     $\langle$\textit{MK}9$\rangle$ & Interaktionen müssen verschiedenen Kategorien zugeteilt werden können.\\
     $\langle$\textit{MK}10$\rangle$ & Das Produkt muss über ein Statistik-\Gls{Dashboard} verfügen.\\
     $\langle$\textit{MK}11$\rangle$ & Der Nutzer muss Statistiken über einzelne Schüler sowie ganze Kurse einsehen können.\\
     $\langle$\textit{MK}12$\rangle$ & Es muss eine serverseitige Speicherung und Synchronisation der Daten stattfinden.\\
     $\langle$\textit{MK}13$\rangle$ & Interaktionsaufzeichnungen müssen exportiert werden können.\\
     $\langle$\textit{MK}14$\rangle$ & Das Produkt muss einen  \Gls{System Usability Scale}-Score von mindestens 68 haben und damit in der Einfachheit der Bedienung als ''gut'' gelten.\\
     $\langle$\textit{MK}15$\rangle$ & Das Produkt muss zwischen Benutzergruppen unterscheiden können. Insbesondere muss es \Gls{Administrator}en geben können. \\

     
\end{tabular}

\subsection{Kannkriterien}
\label{list:Kannkriterien}

\begin{tabular}{p{1.1cm}p{12.9cm}}
     $\langle$\textit{KK}1$\rangle$ &  Schülerprofile und Kurse können mit anderen \Gls{Account}s geteilt werden.\\
     $\langle$\textit{KK}2$\rangle$ & Die Sitzordnung in Kursen kann in der \Gls{Desktop}-Version angepasst werden.\\
     $\langle$\textit{KK}3$\rangle$ & Es können mehrere Sitzordnungen für einen Kurs angelegt werden.\\
     $\langle$\textit{KK}4$\rangle$ & Es kann ein Unterrichtsfach zu einem Kurs hinzugefügt werden.\\
     $\langle$\textit{KK}5$\rangle$ & Durch eine \Gls{undo}-Funktion kann eine Interaktion rückgängig gemacht werden.\\
     $\langle$\textit{KK}6$\rangle$ & Die Qualität einer Interaktion kann angegeben werden.\\
     $\langle$\textit{KK}7$\rangle$ & Der Benutzer kann eigene Kategorien für Interaktionen erstellen.\\
     $\langle$\textit{KK}8$\rangle$ & \Gls{Live}-Statistiken können mithilfe von verschiedenen Farben angezeigt werden, in denen besonders aktive oder passive Schüler hervorgehoben werden.\\
\end{tabular}

\subsection{Abgrenzungskriterien}
\label{list:Abgrenzungskriterien}

\begin{tabular}{p{1.1cm}p{12.9cm}}
     $\langle$\textit{AK}1$\rangle$ & Das Produkt ist ausgelegt, um Lehrende zu unterstützen. Funktionen aus Schülerperspektive wie bspw. eine Dateneinsicht werden nicht unterstützt.\\
        $\langle$\textit{AK}2$\rangle$ & Die graphische Benutzeroberfläche (\Gls{GUI}) ist nicht personalisierbar. Es können weder Informationen ausgeblendet werden noch zusätzliche Informationen angezeigt werden.\\
        $\langle$\textit{AK}3$\rangle$ & Der Benutzer kann keine außer den vordefinierten Statistiken anzeigen lassen.\\
     $\langle$\textit{AK}4$\rangle$ & Interaktionen sind abseits der \Gls{undo}-Funktion nicht editierbar.\\
     $\langle$\textit{AK}5$\rangle$ & Es gibt keine Umkehrung der \Gls{undo}-Funktion.\\
     $\langle$\textit{AK}6$\rangle$ & Es werden keine weiteren Sprachen außer der deutschen Sprache unterstützt. \\
     $\langle$\textit{AK}7$\rangle$ & Das Layout ist nicht farblich anpassbar. Es gibt keine \Gls{Themes}.\\
\end{tabular}



\section{Produkteinsatz}
    \subsection{Anwendungsbereich}
    Das Produkt dient der Erfassung von Interaktionen von Schülergruppen und um deren einfache und schnelle Dokumentation. Es soll damit eine objektive und gerechte Einschätzung der Leistungen der Schüler ermöglicht werden. Ein Lehrer kann Interaktionen innerhalb seiner Schülergruppen dokumentieren und diese in Form von Interaktionskarten visualisieren. Zusätzliche Statistiken ermöglichen eine einfache Analyse von Interaktionen in einer Lernsituation.
    
    \subsection{Zielgruppen}
    Das Produkt ist für Lehrerkräfte ausgelegt, kann aber auch für andere miteinander kommunizierende Gruppen verwendet werden. Das Produkt benötigt keine Vorkenntnisse und kann somit von jedem unabhängig von Alter und Kenntnissen verwendet werden.
    
    \subsection{Betriebsbedingungen}
    Folgende Voraussetzungen müssen erfüllt sein:

    \singlespacing
    \begin{itemize}
        \item Benutzung eines Gerätes mit ausreichender Stromversorgung
        \item Nutzung von einem Progressive-Web-App (\Gls{PWA}) fähigen \Gls{Chromium}-\Gls{Browser} oder des \Gls{Safari}-Browsers
        \item bestehende Internetverbindungen während der \Gls{Synchronisation}, mindestens jedoch alle 30 Tage
        \item ein Gerät mit einer \Gls{Display}größe von mindestens 640 Pixel x 1136 Pixel
        \item Nutzung auf \Gls{Android} oder \gls{iOS} Mobiltelefonen bzw. Tablets oder im \Gls{Desktop}-\Gls{Browser}
        \item mindestens 100MB freier Speicherplatz
    \end{itemize}
    \onehalfspacing

\include{Produktübersicht.tex}

\section{Produktfunktionen}
In diesem Kapitel werden die Funktion des Produkts genauer beschrieben. Bei der Produktfunktionalität wird zwischen den Basis-Funktionen und den Erweiterten Funktionen unterschieden. Die Basis-Funktionen stellen die grundlegenden Funktionalitäten des Produkts dar. Sie sind notwendig, um die Musskriterien\textsuperscript{\ref{list:Musskriterien}} zu erfüllen. Die erweiterten Funktionen ergänzen die Anwendung entsprechend der Kannkriterien\textsuperscript{\ref{list:Kannkriterien}}.

\subsection{Basis-Funktionen}
Die folgende Tabelle gibt einen Überblick über die Basis-Funktionen. Alle Funktionen werden in \ref{section:beschreibungen} genauer beschrieben.\\

\begin{table}[!ht]
    \centering
        \begin{tabular}{p{1cm}|p{9.5cm}|p{3cm}}
                \textbf{\sffamily{Nr.}} & \textbf{\sffamily{Funktion}} & \textbf{\sffamily{Kriterium}}\textsuperscript{\ref{list:Musskriterien}}\\
            \hline
            \hline
                $\langle$\textit F010$\rangle$ & Anzeigen des \Gls{Login}-Bildschirms & $\langle$\textit{MK}1$\rangle$\\
            \hline
                $\langle$\textit F020$\rangle$ & \Gls{Account} anlegen und einloggen & $\langle$\textit{MK}1$\rangle$,$\langle$\textit{MK}12$\rangle$\\
            \hline
                $\langle$\textit F030$\rangle$ & Schüler anzeigen & $\langle$\textit{MK}2$\rangle$\\
            \hline
                $\langle$\textit F040$\rangle$ & Neuen Schüler anlegen & $\langle$\textit{MK}2$\rangle$\\
            \hline
                $\langle$\textit F050$\rangle$ & Schülerstatistiken ansehen & $\langle$\textit{MK}2$\rangle$,$\langle$\textit{MK}10$\rangle$,\newline$\langle$\textit{MK}11$\rangle$,$\langle$\textit{MK}12$\rangle$\\
            \hline
                $\langle$\textit F060$\rangle$ & Kurse anzeigen & $\langle$\textit{MK}3$\rangle$\\
            \hline
                $\langle$\textit F070$\rangle$ & Neuen Kurs anlegen & $\langle$\textit{MK}3$\rangle$\\
            \hline
                $\langle$\textit F080$\rangle$ & Kursstatistiken ansehen & $\langle$\textit{MK}3$\rangle$,$\langle$\textit{MK}10$\rangle$,\newline$\langle$\textit{MK}11$\rangle$,$\langle$\textit{MK}12$\rangle$\\
            \hline
                $\langle$\textit F090$\rangle$ & Schüler in einem Kurs anzeigen & $\langle$\textit{MK}3$\rangle$,$\langle$\textit{MK}4$\rangle$\\
            \hline
                $\langle$\textit F100$\rangle$ & Schüler zu einem Kurs hinzufügen & $\langle$\textit{MK}3$\rangle$,$\langle$\textit{MK}4$\rangle$\\
            \hline
                $\langle$\textit F110$\rangle$ & Schüler aus einem Kurs entfernen & $\langle$\textit{MK}3$\rangle$,$\langle$\textit{MK}4$\rangle$\\
            \hline
                $\langle$\textit F120$\rangle$ & Neue Sitzung erstellen & $\langle$\textit{MK}5$\rangle$,$\langle$\textit{MK}6$\rangle$\\
            \hline
                $\langle$\textit F130$\rangle$ & Interaktion aufzeichnen & $\langle$\textit{MK}5$\rangle$,$\langle$\textit{MK}6$\rangle$,\newline$\langle$\textit{MK}7$\rangle$,$\langle$\textit{MK}8$\rangle$\\
            \hline
                $\langle$\textit F140$\rangle$ & Kategorie für Interaktion auswählen & $\langle$\textit{MK}5$\rangle$,$\langle$\textit{MK}9$\rangle$\\
            \hline
                $\langle$\textit F150$\rangle$ & Interaktionskarten speichern & $\langle$\textit{MK}5$\rangle$,$\langle$\textit{MK}12$\rangle$\\
            \hline
                $\langle$\textit F160$\rangle$ & Interaktionskarten ansehen & $\langle$\textit{MK}5$\rangle$,$\langle$\textit{MK}6$\rangle$,\newline$\langle$\textit{MK}7$\rangle$,$\langle$\textit{MK}12$\rangle$\\
            \hline
                $\langle$\textit F170$\rangle$ & Sitzungsstatistiken ansehen & $\langle$\textit{MK}5$\rangle$,$\langle$\textit{MK}10$\rangle$,\newline$\langle$\textit{MK}12$\rangle$\\
            \hline
                $\langle$\textit F180$\rangle$ & Interaktionskarten exportieren & $\langle$\textit{MK}5$\rangle$,$\langle$\textit{MK}13$\rangle$\\
            \hline
        \end{tabular}
    \caption{Basis-Funktionen}
    \label{table:Basis-Funktionen}
\end{table}

\newpage
\subsection{Erweiterte Funktionen}
Die folgende Tabelle gibt einen Überblick über die erweiterten Funktionen. Alle Funktionen werden in \ref{section:beschreibungen} genauer beschrieben.\\

\begin{table}[!ht]
    \centering
        \begin{tabular}{p{1cm}|p{9.5cm}|p{3cm}}
                \textbf{\sffamily{Nr.}} & \textbf{\sffamily{Funktion}} & \textbf{\sffamily{Kriterium}}\textsuperscript{\ref{list:Kannkriterien}}\\
            \hline
            \hline
                $\langle$F190$\rangle$ & Schüler teilen & $\langle$\textit{KK}1$\rangle$,$\langle$\textit{MK}1$\rangle$,\newline$\langle$\textit{MK}2$\rangle$\\
            \hline
                $\langle$F200$\rangle$ & Kurse teilen  & $\langle$\textit{KK}1$\rangle$,$\langle$\textit{MK}1$\rangle$,\newline$\langle$\textit{MK}3$\rangle$\\
            \hline
                $\langle$F210$\rangle$ & Sitzordnung für Kurse anlegen & $\langle$\textit{KK}2$\rangle$,$\langle$\textit{KK}3$\rangle$,\newline$\langle$\textit{MK}3$\rangle$\\
            \hline
                $\langle$F220$\rangle$ & Unterrichtsfach zu einem Kurs hinzufügen &  $\langle$\textit{KK}4$\rangle$,$\langle$\textit{MK}3$\rangle$\\
            \hline
                $\langle$F230$\rangle$ & Interaktion rückgängig machen mit \Gls{undo}  &  $\langle$\textit{KK}5$\rangle$,$\langle$\textit{MK}5$\rangle$\\
            \hline
                $\langle$F240$\rangle$ & Qualität bei Interaktion angeben &  $\langle$\textit{KK}4$\rangle$,$\langle$\textit{MK}5$\rangle$\\
            \hline
                $\langle$F250$\rangle$ & Eigene Kategorie für Interaktionen erstellen &  $\langle$\textit{KK}7$\rangle$,$\langle$\textit{MK}5$\rangle$,\newline$\langle$\textit{MK}9$\rangle$\\
            \hline
        \end{tabular}
    \caption{Erweiterte Funktionen}
    \label{table:Erweiterte Funktionen}
\end{table}

\newenvironment{beschreibung}{%
  \parskip6pt \parindent0pt \raggedright
  \def\lititem{\hangindent=0.95cm \hangafter1}}{%
  \par\ignorespaces}

\onehalfspacing
\subsection{Funktionsbeschreibungen}
\label{section:beschreibungen}

\textbf{\large \sffamily {Anzeigen des \Gls{Login}-Bildschirms}} $\langle$\textit F10$\rangle$

\begin{beschreibung}
    \lititem{\textbf{\sffamily Anwendungsfall:} Der Nutzer öffnet die Anwendung.\\}
    \lititem{\textbf{\sffamily Anforderung:} MK1\\}
    \lititem{\textbf{\sffamily Ziel:} Gibt dem Nutzer die Möglichkeit sich anzumelden bzw. zu registrieren.\\}
    \lititem{\textbf{\sffamily Vorbedingung:} -\\}
    \lititem{\textbf{\sffamily Nachbedingung:} Der \Gls{Login}-Bildschirm wird angezeigt.\\}
    \lititem{\textbf{\sffamily Akteure:} Nutzer, Server\\}
    \lititem{\textbf{\sffamily Auslösendes Ereignis:} Die Anwendung wird geöffnet.\\}
    \lititem{\textbf{\sffamily Beschreibung:}\\}

    \singlespacing
\begin{enumerate}
    \item Die Anwendung wird geöffnet.
    \item \Gls{Login}-Bildschirm wird angezeigt.
    \item \Gls{Login} ist für die Nutzung der Anwendung erforderlich.
\end{enumerate}
\newpage
\end{beschreibung}
\onehalfspacing


\textbf{\large \sffamily{\Gls{Account} anlegen und einloggen}} $\langle$\textit F20$\rangle$

\begin{beschreibung}
    \lititem{\textbf{\sffamily Anwendungsfall:} Der Nutzer möchte sich anmelden.\\}
    \lititem{\textbf{\sffamily Anforderung:} MK1, MK12\\}
    \lititem{\textbf{\sffamily Ziel:} Der Nutzer legt einen neuen \Gls{Account} an und kann sich einloggen.\\}
    \lititem{\textbf{\sffamily Vorbedingung:} Der \Gls{Login}-Bildschirm wird angezeigt.\\}
    \lititem{\textbf{\sffamily Nachbedingung:} Erfolgreicher \Gls{Login}. Die Startseite wird angezeigt.\\}
    \lititem{\textbf{\sffamily Akteure:} Nutzer, Server\\}
    \lititem{\textbf{\sffamily Auslösendes Ereignis:} Öffnen der Anwendung, Drücken des \Gls{Login}-Buttons.\\}
    \lititem{\textbf{\sffamily Beschreibung:}\\}

    \singlespacing
\begin{enumerate}
    \item \Gls{Login}-Bildschirm wird angezeigt.
    \item Eingabe von Name, E-Mail und Passwort.
    \item Die Erstregistrierung muss von einem \Gls{Administrator} bestätigt werden.
\end{enumerate}
\vspace{1cm}
\end{beschreibung}
\onehalfspacing


\textbf{\large \sffamily{Schüler anzeigen}} $\langle$\textit F30$\rangle$

\begin{beschreibung}
    \lititem{\textbf{\sffamily Anwendungsfall:} Der Nutzer möchte die bereits angelegten Schüler verwalten.\\}
    \lititem{\textbf{\sffamily Anforderung:} MK2\\}
    \lititem{\textbf{\sffamily Ziel:} Anzeigen der Liste an Schülern.\\}
    \lititem{\textbf{\sffamily Vorbedingung:} Der Nutzer befindet sich auf der Startseite.\\}
    \lititem{\textbf{\sffamily Nachbedingung:} Die Schülerbibliothek wird geöffnet.\\}
    \lititem{\textbf{\sffamily Akteure:} Nutzer\\}
    \lititem{\textbf{\sffamily Auslösendes Ereignis:} Drücken des Buttons zum Öffnen der Schülerbibliothek.\\}
    \lititem{\textbf{\sffamily Beschreibung:}\\}

    \singlespacing
\begin{enumerate}
    \item Schülerbibliothek wird geöffnet.
    \item Liste der bereits angelegten Schüler wird angezeigt.
    \item Nutzer kann Schüler verwalten (löschen, hinzufügen und editieren).
    \item Durch Drücken auf einen Schüler werden weitere Informationen sowie Statistiken ($\langle$\textit F50$\rangle$) über ihn angezeigt.
\end{enumerate}
\newpage
\end{beschreibung}
\onehalfspacing


\textbf{\large \sffamily{Neuen Schüler anlegen}} $\langle$\textit F40$\rangle$

\begin{beschreibung}
    \lititem{\textbf{\sffamily Anwendungsfall:} Der Nutzer möchte einen neuen Schüler anlegen.\\}
    \lititem{\textbf{\sffamily Anforderung:} MK2\\}
    \lititem{\textbf{\sffamily Ziel:} Neuer Schüler wird erstellt und zur Schülerliste hinzugefügt.\\}
    \lititem{\textbf{\sffamily Vorbedingung:} Die Schülerbibliothek ist geöffnet.\\}
    \lititem{\textbf{\sffamily Nachbedingung:} Neu erstellter Schüler wird in der Schülerliste angezeigt.\\}
    \lititem{\textbf{\sffamily Akteure:} Nutzer\\}
    \lititem{\textbf{\sffamily Auslösendes Ereignis:} Drücken auf ''+''-Button in der Schülerliste.\\}
    \lititem{\textbf{\sffamily Beschreibung:}\\}

    \singlespacing
\begin{enumerate}
    \item Drücken auf ''+''-Button.
    \item Name des Schülers eingeben.
    \item (Optional) Der Nutzer kann den neu erstellten Nutzer direkt zu einem Kurs hinzufügen. ($\langle$\textit F90$\rangle$)
\end{enumerate}
\vspace{1cm}
\end{beschreibung}
\onehalfspacing


\textbf{\large \sffamily{Schülerstatistiken ansehen}} $\langle$\textit F50$\rangle$

\begin{beschreibung}
    \lititem{\textbf{\sffamily Anwendungsfall:} Der Nutzer möchte Statistiken über einen Schüler ansehen.\\}
    \lititem{\textbf{\sffamily Anforderung:} MK2, MK10, MK11, MK12\\}
    \lititem{\textbf{\sffamily Ziel:} Einsehen der Statistiken eines Schülers.\\}
    \lititem{\textbf{\sffamily Vorbedingung:} Die Schülerbibliothek ist geöffnet.\\}
    \lititem{\textbf{\sffamily Nachbedingung:} Die Statistiken über den ausgewählten Schüler werden angezeigt.\\}
    \lititem{\textbf{\sffamily Akteure:} Nutzer, Server\\}
    \lititem{\textbf{\sffamily Auslösendes Ereignis:} Drücken auf einen Schüler in der Schülerliste.\\}
    \lititem{\textbf{\sffamily Beschreibung:}\\}

    \singlespacing
\begin{enumerate}
    \item Den gewünschten Schüler in der Liste auswählen.
    \item Statistik über den Schüler wird geöffnet.
    \item Der Nutzer wird informiert, falls noch keine Statistiken zu dem Schüler verfügbar sind.
\end{enumerate}
\newpage
\end{beschreibung}
\onehalfspacing


\textbf{\large \sffamily{Kurse anzeigen}} $\langle$\textit F60$\rangle$

\begin{beschreibung}
    \lititem{\textbf{\sffamily Anwendungsfall:} Der Nutzer möchte die bereits angelegten Kurse verwalten.\\}
    \lititem{\textbf{\sffamily Anforderung:} MK3\\}
    \lititem{\textbf{\sffamily Ziel:} Anzeigen der Liste an Kursen.\\}
    \lititem{\textbf{\sffamily Vorbedingung:} Der Nutzer befindet sich auf der Startseite.\\}
    \lititem{\textbf{\sffamily Nachbedingung:} Die Kursbibliothek wird geöffnet.\\}
    \lititem{\textbf{\sffamily Akteure:} Nutzer\\}
    \lititem{\textbf{\sffamily Auslösendes Ereignis:} Drücken des Buttons zum Öffnen der Kursbibliothek.\\}
    \lititem{\textbf{\sffamily Beschreibung:}\\}

    \singlespacing
\begin{enumerate}
    \item Kursbibliothek wird geöffnet.
    \item Liste der bereits angelegten Kurse wird angezeigt.
    \item Nutzer kann Kurse verwalten (löschen, hinzufügen, editieren).
    \item Durch Drücken auf einen Kurs werden weitere Informationen sowie Statistiken ($\langle$\textit F80$\rangle$) über ihn angezeigt.
\end{enumerate}
\vspace{1cm}
\end{beschreibung}
\onehalfspacing


\textbf{\large \sffamily{Neuen Kurs anlegen}} $\langle$\textit F70$\rangle$

\begin{beschreibung}
    \lititem{\textbf{\sffamily Anwendungsfall:} Der Nutzer möchte einen neuen Kurs anlegen.\\}
    \lititem{\textbf{\sffamily Anforderung:} MK3\\}
    \lititem{\textbf{\sffamily Ziel:} Neuer Kurs wird erstellt und zur Kursliste hinzugefügt.\\}
    \lititem{\textbf{\sffamily Vorbedingung:} Die Kursbibliothek ist geöffnet.\\}
    \lititem{\textbf{\sffamily Nachbedingung:} Neu erstellter Kurs wird in der Kursliste angezeigt.\\}
    \lititem{\textbf{\sffamily Akteure:} Nutzer\\}
    \lititem{\textbf{\sffamily Auslösendes Ereignis:} Drücken auf ''+''-Button in der Kursliste.\\}
    \lititem{\textbf{\sffamily Beschreibung:}\\}

    \singlespacing
\begin{enumerate}
    \item Drücken auf ''+''-Button.
    \item Name des Kurses eingeben.
    \item (Optional) Der Nutzer kann ein Unterrichtsfach zum Kurs hinzufügen. ($\langle$\textit F220$\rangle$)
\end{enumerate}
\newpage
\end{beschreibung}
\onehalfspacing


\textbf{\large \sffamily{Kursstatistiken ansehen}} $\langle$\textit F80$\rangle$

\begin{beschreibung}
    \lititem{\textbf{\sffamily Anwendungsfall:} Der Nutzer möchte Statistiken über einen Kurs ansehen.\\}
    \lititem{\textbf{\sffamily Anforderung:} MK3, MK10, MK11, MK12\\}
    \lititem{\textbf{\sffamily Ziel:} Einsehen der Statistiken eines Kurses.\\}
    \lititem{\textbf{\sffamily Vorbedingung:} Die Kursbibliothek ist geöffnet.\\}
    \lititem{\textbf{\sffamily Nachbedingung:} Die Statistiken über den ausgewählten Kurs werden angezeigt.\\}
    \lititem{\textbf{\sffamily Akteure:} Nutzer, Server\\}
    \lititem{\textbf{\sffamily Auslösendes Ereignis:} Drücken auf einen Kurs in der Kursliste.\\}
    \lititem{\textbf{\sffamily Beschreibung:}\\}

    \singlespacing
\begin{enumerate}
    \item Den gewünschten Kurs in der Kursliste auswählen.
    \item Statistik über den Kurs wird geöffnet.
    \item Der Nutzer wird informiert, falls noch keine Statistiken zu dem Kurs verfügbar sind. 
\end{enumerate}
\vspace{1cm}
\end{beschreibung}
\onehalfspacing


\textbf{\large \sffamily{Schüler in einem Kurs anzeigen}} $\langle$\textit F90$\rangle$

\begin{beschreibung}
    \lititem{\textbf{\sffamily Anwendungsfall:} Der Nutzer möchte die Schüler in einem Kurs anzeigen.\\}
    \lititem{\textbf{\sffamily Anforderung:} MK3, MK4\\}
    \lititem{\textbf{\sffamily Ziel:} Anzeigen der Liste an Schülern in einem Kurs.\\}
    \lititem{\textbf{\sffamily Vorbedingung:} Ein Kurs wurde in der Kursliste ausgewählt.\\}
    \lititem{\textbf{\sffamily Nachbedingung:} Schülerliste des Kurses wird geöffnet.\\}
    \lititem{\textbf{\sffamily Akteure:} Nutzer\\}
    \lititem{\textbf{\sffamily Auslösendes Ereignis:} Drücken auf Schülerliste im Kurs.\\}
    \lititem{\textbf{\sffamily Beschreibung:}\\}

    \singlespacing
\begin{enumerate}
    \item Die Schülerliste des Kurses wird geöffnet.
    \item Nutzer kann die Schüler des Kurses ansehen und verwalten (hinzufügen, löschen).
    \item Durch Drücken auf einen Schüler werden weitere Informationen sowie Statistiken ($\langle$\textit F50$\rangle$) über ihn angezeigt.
\end{enumerate}
\newpage
\end{beschreibung}
\onehalfspacing


\textbf{\large \sffamily{Schüler zu einem Kurs hinzufügen}} $\langle$\textit F100$\rangle$

\begin{beschreibung}
    \lititem{\textbf{\sffamily Anwendungsfall:} Der Nutzer möchte einen Schüler zu einem Kurs hinzufügen. \\}
    \lititem{\textbf{\sffamily Anforderung:} MK3, MK4\\}
    \lititem{\textbf{\sffamily Ziel:} Ein Schüler wird zu einem Kurs hinzugefügt.\\}
    \lititem{\textbf{\sffamily Vorbedingung:} Die Schülerliste des Kurses ist geöffnet.\\}
    \lititem{\textbf{\sffamily Nachbedingung:} Der hinzugefügte Schüler wird in der Liste angezeigt.\\}
    \lititem{\textbf{\sffamily Akteure:} Nutzer\\}
    \lititem{\textbf{\sffamily Auslösendes Ereignis:} Drücken auf ''+''-Button in der Schülerliste des Kurses.\\}
    \lititem{\textbf{\sffamily Beschreibung:}\\}

    \singlespacing
\begin{enumerate}
    \item Drücken auf ''+''-Button.
    \item Schüler aus Schülerbibliothek auswählen.
    \item Ein Schüler kann zu mehreren Kursen hinzugefügt werden.
\end{enumerate}
\vspace{1cm}
\end{beschreibung}
\onehalfspacing


\textbf{\large \sffamily{Schüler aus Kurs entfernen}} $\langle$\textit F110$\rangle$

\begin{beschreibung}
    \lititem{\textbf{\sffamily Anwendungsfall:} Der Nutzer möchte einen Schüler aus einem Kurs entfernen.\\}
    \lititem{\textbf{\sffamily Anforderung:} MK3, MK4\\}
    \lititem{\textbf{\sffamily Ziel:} Ein Schüler wird aus dem Kurs entfernt.\\}
    \lititem{\textbf{\sffamily Vorbedingung:} Die Schülerliste des Kurses ist geöffnet.\\}
    \lititem{\textbf{\sffamily Nachbedingung:} \\}
    \lititem{\textbf{\sffamily Akteure:} Nutzer\\}
    \lititem{\textbf{\sffamily Auslösendes Ereignis:} Drücken auf Entfernen-Button neben dem Schüler in der Schülerliste des Kurses.\\}
    \lititem{\textbf{\sffamily Beschreibung:}\\}

    \singlespacing
\begin{enumerate}
    \item Der Schüler wird aus dem Kurs entfernt.
    \item Der Schüler existiert weiterhin in der Schülerbibliothek.
\end{enumerate}
\newpage
\end{beschreibung}
\onehalfspacing


\textbf{\large \sffamily{Neue Sitzung erstellen}} $\langle$\textit F120$\rangle$

\begin{beschreibung}
    \lititem{\textbf{\sffamily Anwendungsfall:} Der Nutzer möchte eine neue Sitzung für die Interaktionsaufzeichnung erstellen.\\}
    \lititem{\textbf{\sffamily Anforderung:} MK5, MK5\\}
    \lititem{\textbf{\sffamily Ziel:} Der Nutzer erstellt eine neue Interaktionskarte für den gewählten Kurs.\\}
    \lititem{\textbf{\sffamily Vorbedingung:} Der Schüler befindet sich auf der Startseite.\\}
    \lititem{\textbf{\sffamily Nachbedingung:} Eine neue Sitzung wird erstellt. Die Schüler des Kurses werden angezeigt und der Nutzer kann Interaktionen aufzeichnen.\\}
    \lititem{\textbf{\sffamily Akteure:} Nutzer\\}
    \lititem{\textbf{\sffamily Auslösendes Ereignis:} ''Neue Sitzung''-Button gedrückt.\\}
    \lititem{\textbf{\sffamily Beschreibung:}\\}

    \singlespacing
\begin{enumerate}
    \item Den Kurs für die Interaktionsaufzeichnung wählen.
    \item Eine neue Sitzung wird erstellt.
    \item Interaktionen können aufgezeichnet werden. ($\langle$\textit F130$\rangle$)
\end{enumerate}
\vspace{1cm}
\end{beschreibung}
\onehalfspacing


\textbf{\large \sffamily{Interaktion aufzeichnen}} $\langle$\textit F130$\rangle$

\begin{beschreibung}
    \lititem{\textbf{\sffamily Anwendungsfall:} Der Nutzer möchte eine Interaktion von Schülern aufzeichnen.\\}
    \lititem{\textbf{\sffamily Anforderung:} MK5, MK6, MK7, MK8\\}
    \lititem{\textbf{\sffamily Ziel:} Interaktionen zwischen Schülern sowie zwischen Schülern und Lehrer erfassen.\\}
    \lititem{\textbf{\sffamily Vorbedingung:} Eine neue Sitzung wurde erstellt.\\}
    \lititem{\textbf{\sffamily Nachbedingung:} Ein Fenster für die Kategorieauswahl öffnet sich.\\}
    \lititem{\textbf{\sffamily Akteure:} Nutzer\\}
    \lititem{\textbf{\sffamily Auslösendes Ereignis:} Auf einen Schüler drücken.\\}
    \lititem{\textbf{\sffamily Beschreibung:}\\}

    \singlespacing
\begin{enumerate}
    \item Interaktion zwischen Schülern/Lehrer wird erfasst.
    \item Die Interaktion kann einer Kategorie zugeordnet werden. ($\langle$\textit F140$\rangle$)
\end{enumerate}
\newpage
\end{beschreibung}
\onehalfspacing


\textbf{\large \sffamily{Kategorie für Interaktion auswählen}} $\langle$\textit F140$\rangle$

\begin{beschreibung}
    \lititem{\textbf{\sffamily Anwendungsfall:} Der Nutzer möchte Interaktionen von Schülern bewerten und einer Kategorie zuordnen.\\}
    \lititem{\textbf{\sffamily Anforderung:} MK5, MK9\\}
    \lititem{\textbf{\sffamily Ziel:} Die Interaktion wird einer Kategorie zugeordnet.\\}
    \lititem{\textbf{\sffamily Vorbedingung:} Eine Interaktion wurde erfasst und das Fenster für die Kategorieauswahl ist geöffnet.\\}
    \lititem{\textbf{\sffamily Nachbedingung:} Die Kategorie wurde ausgewählt. Die Schüler in der Sitzung werden wieder angezeigt und es können weitere Interaktionen aufgezeichnet werden.\\}
    \lititem{\textbf{\sffamily Akteure:} Nutzer\\}
    \lititem{\textbf{\sffamily Auslösendes Ereignis:} Interaktion zwischen Schülern wurde erfasst.\\}
    \lititem{\textbf{\sffamily Beschreibung:}\\}

    \singlespacing
\begin{enumerate}
    \item Die Interaktion wird einer Kategorie zugeordnet.
    \item Der Nutzer kann ggf. eigene Kategorien definieren. ($\langle$\textit F250$\rangle$)
\end{enumerate}
\vspace{1cm}
\end{beschreibung}
\onehalfspacing


\textbf{\large \sffamily{Interaktionskarten speichern}} $\langle$\textit F150$\rangle$

\begin{beschreibung}
    \lititem{\textbf{\sffamily Anwendungsfall:} Der Nutzer möchte die Interaktionskarte einer Sitzung speichern.\\}
    \lititem{\textbf{\sffamily Anforderung:} MK5, MK12\\}
    \lititem{\textbf{\sffamily Ziel:} Die Interaktionskarte wird in den Aufzeichnungen gespeichert und kann jederzeit angesehen werden.\\}
    \lititem{\textbf{\sffamily Vorbedingung:} Eine Interaktionskarte wurde in einer Sitzung erstellt.\\}
    \lititem{\textbf{\sffamily Nachbedingung:} Die Interaktionskarte ist in den Aufzeichnungen einsehbar.\\}
    \lititem{\textbf{\sffamily Akteure:} Nutzer\\}
    \lititem{\textbf{\sffamily Auslösendes Ereignis:} Eine Sitzung wird beendet.\\}
    \lititem{\textbf{\sffamily Beschreibung:}\\}

    \singlespacing
\begin{enumerate}
    \item Die Interaktionskarte wird in den Aufzeichnungen gespeichert.
    \item Interaktionskarten werden bei vorhandener Internetverbindung zwischen Geräten synchronisiert.
    \item (Optional) Der Nutzer kann die erstellte Interaktionskarte auf Wunsch direkt exportieren. ($\langle$\textit F180$\rangle$)
\end{enumerate}
\newpage
\end{beschreibung}
\onehalfspacing


\textbf{\large \sffamily{Interaktionskarten ansehen}} $\langle$\textit F160$\rangle$

\begin{beschreibung}
    \lititem{\textbf{\sffamily Anwendungsfall:} Der Nutzer möchte Interaktionskarten von vorherigen Sitzungen ansehen.\\}
    \lititem{\textbf{\sffamily Anforderung:} MK5, MK6, MK7, MK12\\}
    \lititem{\textbf{\sffamily Ziel:} Interaktionen von vergangenen Sitzungen einsehen.\\}
    \lititem{\textbf{\sffamily Vorbedingung:} Der Nutzer befindet sich auf der Startseite.\\}
    \lititem{\textbf{\sffamily Nachbedingung:} Eine Übersicht über die gespeicherten Interaktionskarten wird geöffnet.\\}
    \lititem{\textbf{\sffamily Akteure:} Nutzer, Server\\}
    \lititem{\textbf{\sffamily Auslösendes Ereignis:} Der Menüpunkt ''Interaktionskarten'' wird gedrückt.\\}
    \lititem{\textbf{\sffamily Beschreibung:}\\}

    \singlespacing
\begin{enumerate}
    \item Übersicht der Interaktionskarten wird geöffnet.
    \item Nutzer kann Interaktionskarten ansehen und löschen.
    \item Durch Drücken auf eine Interaktionskarte werden weitere Informationen sowie die Statistiken der Sitzung ($\langle$\textit F170$\rangle$) angezeigt.
\end{enumerate}
\vspace{1cm}
\end{beschreibung}
\onehalfspacing


\textbf{\large \sffamily{Sitzungsstatistiken ansehen}} $\langle$\textit F170$\rangle$

\begin{beschreibung}
    \lititem{\textbf{\sffamily Anwendungsfall:} Der Nutzer möchte Statistiken zu vergangenen Sitzungen einsehen.\\}
    \lititem{\textbf{\sffamily Anforderung:} MK5, MK10, MK12\\}
    \lititem{\textbf{\sffamily Ziel:} Anzeigen der Statistiken einer vergangenen Sitzung.\\}
    \lititem{\textbf{\sffamily Vorbedingung:} Die Übersicht über die Interaktionskarten ist geöffnet.\\}
    \lititem{\textbf{\sffamily Nachbedingung:} Die Statistiken über die ausgewählte Sitzung werden angezeigt.\\}
    \lititem{\textbf{\sffamily Akteure:} Nutzer, Server\\}
    \lititem{\textbf{\sffamily Auslösendes Ereignis:} Drücken auf eine Interaktionskarte in der Übersicht.\\}
    \lititem{\textbf{\sffamily Beschreibung:}\\}

    \singlespacing
\begin{enumerate}
    \item Interaktionskarte der gewünschten Sitzung auswählen.
    \item Statistik über die Sitzung wird geöffnet.
\end{enumerate}
\newpage
\end{beschreibung}
\onehalfspacing


\textbf{\large \sffamily{Interaktionskarten exportieren}} $\langle$\textit F180$\rangle$

\begin{beschreibung}
    \lititem{\textbf{\sffamily Anwendungsfall:} Der Nutzer möchte Interaktionskarten exportieren, um sie auch ohne Internetverbindung einsehen zu können.\\}
    \lititem{\textbf{\sffamily Anforderung:} MK5, MK13\\}
    \lititem{\textbf{\sffamily Ziel:} Interaktionskarten in einem geeigneten Format exportieren.\\}
    \lititem{\textbf{\sffamily Vorbedingung:} Eine Interaktionskarte wurde in der Übersicht ausgewählt.\\}
    \lititem{\textbf{\sffamily Nachbedingung:} Die Interaktionskarte wurde erfolgreich im festgelegten Format exportiert.\\}
    \lititem{\textbf{\sffamily Akteure:} Nutzer\\}
    \lititem{\textbf{\sffamily Auslösendes Ereignis:} ''Exportieren''-Button wird gedrückt.\\}
    \lititem{\textbf{\sffamily Beschreibung:}\\}

    \singlespacing
\begin{enumerate}
    \item Die ausgewählte Interaktionskarte wird im festgelegten Format exportiert.
    \item Interaktionskarten können auch direkt nach Beenden einer Sitzung exportiert werden.
\end{enumerate}
\vspace{1cm}
\end{beschreibung}
\onehalfspacing


\textbf{\large \sffamily{Schüler teilen}} $\langle$\textit F190$\rangle$

\begin{beschreibung}
    \lititem{\textbf{\sffamily Anwendungsfall:} Der Nutzer möchte Schüler mit anderen \Gls{Account}s (Lehrern) teilen.\\}
    \lititem{\textbf{\sffamily Anforderung:} KK1, MK1, MK2\\}
    \lititem{\textbf{\sffamily Ziel:} Ein Schüler wird mit einem anderen \Gls{Account} geteilt und dort automatisch in der Schülerbibliothek angelegt. Alle Informationen und Statistiken des Schülers werden übernommen.\\}
    \lititem{\textbf{\sffamily Vorbedingung:} Der gewünschte Schüler wurde in der Schülerbibliothek ausgewählt. Die E-Mail des \Gls{Account}s, mit dem der Schüler geteilt werden soll, ist bekannt.\\}
    \lititem{\textbf{\sffamily Nachbedingung:} Der geteilte Schüler wird bei dem anderen \Gls{Account} in der Schülerbibliothek angezeigt\\}
    \lititem{\textbf{\sffamily Akteure:} Nutzer, Server\\}
    \lititem{\textbf{\sffamily Auslösendes Ereignis:} Drücken auf ''Teilen''-Button in Schüleransicht.\\}
    \lititem{\textbf{\sffamily Beschreibung:}\\}

    \singlespacing
\begin{enumerate}
    \item E-Mail des \Gls{Account}s angeben, mit dem der Schüler geteilt werden soll.
    \item Der Schüler wird automatisch angelegt.
\end{enumerate}
\newpage
\end{beschreibung}
\onehalfspacing

\textbf{\large \sffamily{Kurse teilen}} $\langle$\textit F200$\rangle$

\begin{beschreibung}
    \lititem{\textbf{\sffamily Anwendungsfall:} Der Nutzer möchte Kurse mit anderen \Gls{Account}s (Lehrern) teilen.\\}
    \lititem{\textbf{\sffamily Anforderung:} KK1, MK1, MK3\\}
    \lititem{\textbf{\sffamily Ziel:} Ein Kurs wird mit einem anderen \Gls{Account} geteilt und dort automatisch in der Kursbibliothek angelegt. Alle Informationen und Statistiken des Kurses werden übernommen.\\}
    \lititem{\textbf{\sffamily Vorbedingung:} Der gewünschte Kurs wurde in der Kursbibliothek ausgewählt. Die E-Mail des \Gls{Account}s, mit dem der Kurs geteilt werden soll, ist bekannt.\\}
    \lititem{\textbf{\sffamily Nachbedingung:} Der geteilte Kurs wird bei dem anderen \Gls{Account} in der Kursbibliothek angezeigt.\\}
    \lititem{\textbf{\sffamily Akteure:} Nutzer, Server\\}
    \lititem{\textbf{\sffamily Auslösendes Ereignis:} Drücken auf ''Teilen''-Button in der Kursansicht.\\}
    \lititem{\textbf{\sffamily Beschreibung:}\\}

    \singlespacing
\begin{enumerate}
    \item E-Mail des \Gls{Account}s angeben, mit dem der Kurs geteilt werden soll.
    \item Der Kurs wird automatisch angelegt. Befinden sich in dem Kurs Schüler, die noch nicht in der Schülerbibliothek existieren, werden diese ebenfalls angelegt.
\end{enumerate}
\vspace{1cm}
\end{beschreibung}
\onehalfspacing


\textbf{\large \sffamily{Sitzordnung für einen Kurs anlegen}} $\langle$\textit F210$\rangle$

\begin{beschreibung}
    \lititem{\textbf{\sffamily Anwendungsfall:} Der Nutzer möchte eine Sitzordnung für einen Kurs anlegen.\\}
    \lititem{\textbf{\sffamily Anforderung:} KK2, KK3, MK3\\}
    \lititem{\textbf{\sffamily Ziel:} Anlegen einer festen Sitzordnung der Schüler innerhalb eines Kurses.\\}
    \lititem{\textbf{\sffamily Vorbedingung:} Der gewünschte Kurs wurde in der Kursbibliothek ausgewählt. Der Nutzer befindet sich in der Desktop-Ansicht der Anwendung.\\}
    \lititem{\textbf{\sffamily Nachbedingung:} Die Sitzordnung des Kurses wird in der Desktop-Ansicht von Sitzungen angezeigt.\\}
    \lititem{\textbf{\sffamily Akteure:} Nutzer\\}
    \lititem{\textbf{\sffamily Auslösendes Ereignis:} Drücken auf ''Sitzordnung erstellen''-Button in der Kursansicht.\\}
    \lititem{\textbf{\sffamily Beschreibung:}\\}

    \singlespacing
\begin{enumerate}
    \item Auswählen eines Kurses in der Kursbibliothek.
    \item Drücken auf ''Sitzordnung erstellen''-Button.
\end{enumerate}
\newpage
\end{beschreibung}
\onehalfspacing


\textbf{\large \sffamily{Unterrichtsfach zu einem Kurs hinzufügen}} $\langle$\textit F220$\rangle$

\begin{beschreibung} 
    \lititem{\textbf{\sffamily Anwendungsfall:} Der Nutzer möchte ein Unterrichtsfach zu einem Kurs hinzufügen.\\}
    \lititem{\textbf{\sffamily Anforderung:} KK4, MK3\\}
    \lititem{\textbf{\sffamily Ziel:} Dem Kurs wird ein Unterrichtsfach zugeordnet.\\}
    \lititem{\textbf{\sffamily Vorbedingung:} Der gewünschte Kurs wurde in der Kursbibliothek ausgewählt.\\}
    \lititem{\textbf{\sffamily Nachbedingung:} Das Unterrichtsfach wird in der Kursansicht angezeigt.\\}
    \lititem{\textbf{\sffamily Akteure:} Nutzer\\}
    \lititem{\textbf{\sffamily Auslösendes Ereignis:} Drücken auf ''Fach hinzufügen''-Button in der Kursansicht.\\}
    \lititem{\textbf{\sffamily Beschreibung:}\\}

    \singlespacing
\begin{enumerate}
    \item Auswählen eines Kurses in der Kursbibliothek.
    \item Drücken auf ''Fach hinzufügen''-Button.
    \item Kurse können in der Kursbibliothek nach dem Fach sortiert werden.
\end{enumerate}
\vspace{1cm}
\end{beschreibung}
\onehalfspacing

\textbf{\large \sffamily{Interaktion rückgängig machen mit undo}} $\langle$\textit F230$\rangle$

\begin{beschreibung}
    \lititem{\textbf{\sffamily Anwendungsfall:} Der Nutzer möchte eine eingegebene Interaktion zwischen Schülern rückgängig machen. Dies ist beispielsweise der Fall, wenn der Nutzer eine falsche Eingabe getätigt hat.\\}
    \lititem{\textbf{\sffamily Anforderung:} KK5, MK5\\}
    \lititem{\textbf{\sffamily Ziel:} Löschen einer eingegebenen Interaktion aus der Interaktionskarte.\\}
    \lititem{\textbf{\sffamily Vorbedingung:} Eine Interaktion wurde aufgezeichnet.\\}
    \lititem{\textbf{\sffamily Nachbedingung:} Die Interaktion wird aus der Interaktionskarte gelöscht.\\}
    \lititem{\textbf{\sffamily Akteure:} Nutzer\\}
    \lititem{\textbf{\sffamily Auslösendes Ereignis:} Drücken auf ''\Gls{undo}''-Button.\\}
    \lititem{\textbf{\sffamily Beschreibung:}\\}
    
    \singlespacing
\begin{enumerate}
    \item Die zuletzt eingegebene Interaktion wird rückgängig gemacht.
    \item Die Interaktion wird aus der Interaktionskarte gelöscht und wird in der Statistik der Sitzung nicht beachtet.
\end{enumerate}
\newpage
\end{beschreibung}
\onehalfspacing


\textbf{\large \sffamily{Qualität bei Interaktion angeben}} $\langle$\textit F240$\rangle$

\begin{beschreibung}
    \lititem{\textbf{\sffamily Anwendungsfall:} Der Nutzer möchte Interaktionen von Schülern in Hinsicht auf Qualität bewerten (zum Beispiel die Komplexität einer Antwort).\\}
    \lititem{\textbf{\sffamily Anforderung:} KK4, MK5\\}
    \lititem{\textbf{\sffamily Ziel:} Bewertung der Interaktion eines Schülers.\\}
    \lititem{\textbf{\sffamily Vorbedingung:} Eine Interaktion wurde erfasst und das Fenster für die Qualitätsangabe ist geöffnet.\\}
    \lititem{\textbf{\sffamily Nachbedingung:} Die Qualität wurde angegeben. Die Schüler in der Sitzung werden wieder angezeigt und es können weitere Interaktionen aufgezeichnet werden\\}
    \lititem{\textbf{\sffamily Akteure:} Nutzer\\}
    \lititem{\textbf{\sffamily Auslösendes Ereignis:} Eine Interaktion wurde erfasst.\\}
    \lititem{\textbf{\sffamily Beschreibung:}\\}

    \singlespacing
\begin{enumerate}
    \item Nach Erfassen einer Interaktion kann die Qualität angegeben werden.
    \item Die Qualität wird mit einem Sternesystem bewertet.
\end{enumerate}
\vspace{1cm}
\end{beschreibung}
\onehalfspacing


\textbf{\large \sffamily{Eigene Kategorie für Interaktionen erstellen}} $\langle$\textit F250$\rangle$

\begin{beschreibung}
    \lititem{\textbf{\sffamily Anwendungsfall:} Der Nutzer möchte eigene Kategorien für Interaktionen erstellen.\\}
    \lititem{\textbf{\sffamily Anforderung:} KK7, MK5, MK9\\}
    \lititem{\textbf{\sffamily Ziel:} Eine benutzerdefinierte Interaktionskategorie wird erstellt.\\}
    \lititem{\textbf{\sffamily Vorbedingung:} Eine Interaktion wurde erfasst und das Fenster für die Kategorieauswahl ist geöffnet.\\}
    \lititem{\textbf{\sffamily Nachbedingung:} Die neu erstellte Kategorie wurde zur Kategorieauswahl hinzugefügt.\\}
    \lititem{\textbf{\sffamily Akteure:} Nutzer\\}
    \lititem{\textbf{\sffamily Auslösendes Ereignis:} Drücken auf ''+''-Button bei Kategorieauswahl.\\}
    \lititem{\textbf{\sffamily Beschreibung:}\\}

    \singlespacing
\begin{enumerate}
    \item Nach Erfassen einer Interaktion kann bei der Auswahl einer Kategorie eine eigene erstellt werden.
    \item Die neu erstellte Kategorie wird gespeichert und ist von nun an bei jeder Interaktion verfügbar.
\end{enumerate}
\end{beschreibung}
\onehalfspacing

\section{Produktdaten}
Für die Nutzung des Produkts werden Daten lokal und auf einem \Gls{Server} gespeichert. Die \Gls{Login}-Daten werden immer auf dem \Gls{Server} gespeichert und lokal für 30 Tage gehalten. Die weiteren Daten werden bei Eingabe erst lokal gespeichert und bei Zugang zum Internet mit dem \Gls{Server} synchronisiert.

\paragraph{\Gls{Server}daten}

    \singlespacing
    \begin{itemize}
            \item $\langle$\textit D10$\rangle$ Anmeldedaten der Nutzer
            \item  $\langle$\textit D20$\rangle$ Profildaten der Nutzer
            \begin{itemize}
                \item Kurse
                \item Schüler
                \item Sitzordnungen
                \item Unterrichtsfächer
                \item Raumordnungen
                \item Interaktionskarten
            \end{itemize}
            \item $\langle$\textit D30$\rangle$ Schülerstatistiken
            \item $\langle$\textit D40$\rangle$ Kursstatistiken
           \item $\langle$\textit D50$\rangle$ Sitzungsstatistiken
        \end{itemize}
        \onehalfspacing

\paragraph{Lokale Daten} $~$ 

Wird das Produkt offline genutzt, werden alle Änderungen bis zur nächsten Synchronisation lokal gespeichert.

\section{Nichtfunktionale Anforderungen}

\subsection{Kompatibilität}

\begin{tabular}{p{1cm}p{13cm}}
     $\langle$\textit K10$\rangle$ & Das Produkt muss als Progressive Web App (\Gls{PWA}) implementiert werden und mit \Gls{PWA} fähigen \Gls{Chromium}-\Gls{Browser}n sowie \Gls{Safari} installierbar sein.\\
     $\langle$\textit K20$\rangle$ & Das Produkt muss nach der Installation auch offline im vollen Umfang verwendet werden können, mit Ausnahme der \Gls{Synchronisation} vom Produktzustand mit dem \Gls{Server}.\\
     $\langle$\textit K20$\rangle$ & Das Produkt muss - wo möglich - auf Geräten mit \Gls{Touchscreen} die übliche \Gls{Swipe}-Navigierungsgesten blockieren, da diese bei der vorgesehenen Bedienung stören könnten.\\
     $\langle$\textit K30$\rangle$ & Die Benutzeroberfläche des Produkts muss auf allen \Gls{Display}s mit einer Größe von mindestens 640 Pixel × 1136 Pixel korrekt darstellbar sein.\\
     $\langle$\textit K40$\rangle$ & Die Benutzeroberfläche des Produkts muss für die Bedienung auf \Gls{Android} und \gls{iOS} Mobiltelefonen sowie auf Tablets oder auf \Gls{Desktop}-Browsern angepasst sein.\\
     $\langle$\textit K50$\rangle$ & Die \Gls{PWA} des Produkts muss ohne Nutzerdaten nicht mehr als 100MB Speicherplatz beanspruchen.\\
\end{tabular}

\subsection{Benutzbarkeit}

\begin{tabular}{p{1cm}p{13cm}}
     $\langle$\textit B10$\rangle$ & Das Produkt muss einen \Gls{System Usability Scale}-Score von mindestens 68 erreichen.\\
     $\langle$\textit B20$\rangle$ & Das Produkt muss den Nutzer auf Fehler in seinen Eingaben hinweisen.\\
     $\langle$\textit B30$\rangle$ & Das Produkt muss dem Nutzer zu erkennen geben, wenn eine \Gls{Synchronisation} mit dem \Gls{Server} nicht möglich ist.\\
     $\langle$\textit B40$\rangle$ & Das Produkt muss für jede Installation nach einmaliger Anmeldung für mindestens 30 Tage ohne erneute Anmeldung benutzbar sein und bei vorhandener Internetverbindung mit dem \Gls{Server} synchronisieren können, sofern der Nutzer keine Aktionen unternimmt, um dies explizit zu unterbinden.\\
\end{tabular}

\subsection{Fehlertoleranz und Stabilität}

\begin{tabular}{p{1cm}p{13cm}}
     $\langle$\textit{FS}10$\rangle$ & Das Produkt muss nach eventuellem Absturz alle beendeten Eingaben des Nutzers vor dem Absturz wiederherstellen können.\\
     $\langle$\textit{FS}20$\rangle$ & Das Produkt muss bei \Gls{Synchronisation}skonflikten mit dem \Gls{Server} immer diejenige Änderung beibehalten, die zuletzt vom Nutzer getätigt wurde.\\
\end{tabular}

\subsection{Sicherheit und Datenschutz}

\begin{tabular}{p{1cm}p{13cm}}
     $\langle$\textit{SD}10$\rangle$ & Die Nutzerdaten vom Benutzer müssen mit einem \Gls{Login} bestehend aus einem Benutzernamen und Passwort gesichert sein.\\
     $\langle$\textit{SD}20$\rangle$ & Die Nutzerdaten dürfen nur von dem entsprechenden Benutzer sowie \Gls{Administrator}en eingesehen und verändert werden.\\
     $\langle$\textit{SD}30$\rangle$ & Passwörter dürfen auf dem \Gls{Server} nie in Klartext gespeichert werden, sondern müssen mit einem als sicher eingestuften Verfahren \gls{gehasht} werden.\\
     $\langle$\textit{SD}40$\rangle$ & Kommunikation mit dem \Gls{Server} muss via \Gls{SSL} verschlüsselten \Gls{HTTP}-Anfragen erfolgen.\\
     $\langle$\textit{SD}50$\rangle$ & Das Produkt muss ein Impressum sowie eine Datenschutzerklärung bereitstellen.\\
\end{tabular}

\subsection{Qualität}

\begin{tabular}{p{1cm}p{13cm}}
     $\langle$\textit Q10$\rangle$ & Der Programmcode muss selbsterklärend sein, oder wo dies nicht möglich ist, gut kommentiert sein.\\
     $\langle$\textit Q20$\rangle$ & Das Produkt muss eine Dokumentation bereitstellen.\\
     $\langle$\textit Q30$\rangle$ & Das Produkt muss ohne große Änderungen von existierendem Code erweitert werden können.\\
     $\langle$\textit Q40$\rangle$ & Die Sprache des Produkts muss Deutsch sein.\\
     $\langle$\textit Q50$\rangle$ & Die Korrektheit der Software muss durch kontinuierliches Testen sichergestellt werden.\\
\end{tabular}











\include{Benutzeroberfläche Schnittstellen.tex}

\section{Technische Produktumgebung}
    \subsection{Hardware}
        \begin{itemize}
            \item (mobiles) Gerät mit Internetzugang.
            \item 100MB freier Speicher auf dem Gerät
        \end{itemize}
    \subsection{Software}
        \paragraph{Frontend}
            Das \Gls{Frontend} ist eine \Gls{PWA} und ist über einen beliebigen \Gls{Browser} zu erreichen.
            Es basiert auf dem  \Gls{Framework} \Gls{Vue.js} und ist in \Gls{TypeScript} und \Gls{HTML5} geschrieben.
            
        \paragraph{Backend}
            Das \Gls{Server}-\Gls{Backend} läuft auf einer virtuellen Maschine (\Gls{VM}) auf einem \Gls{Server}.
            Auf der \Gls{VM} läuft Linux Ubuntu 20.04 mit der Software \Gls{Spring-Boot} als \Gls{Backend}-\Gls{Framework}.
            Das \Gls{Backend} ist in \Gls{Java} 17 geschrieben.
            Die Daten werden in einer \Gls{MySQL}-\Gls{Datenbank} gespeichert.
    \subsection{Schnittstellen}
        Die Benutzer\gls{Schnittstelle} wird über eine \Gls{GUI} zur Verfügung gestellt.
        Das \Gls{Frontend} kommuniziert über eine \Gls{Rest}-\Gls{API} mit dem \Gls{Backend}.
        
            


\section{Testfälle und Testszenarien}
In diesem Kapitel werden die Testfälle und Testfallszenarien definiert. Sie helfen dabei, die Korrektheit der Produktfunktionalität sicherzustellen.

\subsection{Testfälle}
\label{section:testfaelle}
Jede Produktfunktion wird durch einen entsprechenden Testfall abgedeckt. Wie auch bei den Funktionen wird zwischen Basis-Testfällen und erweiterten Testfällen unterschieden.
    
    \paragraph{\large Basis-Testfälle} $~$ \\
        
    \begin{table}[h!]
        \centering
        \begin{tabular}{p{1.5cm}|p{9cm}|p{3cm}}
            \textbf{\sffamily{Nr.}} & \textbf{\sffamily{Beschreibung}} & \textbf{\sffamily{Funktion}}\textsuperscript{\ref{table:Basis-Funktionen}}\\
            \hline
            \hline
                $\langle T010 \rangle$ & Aufrufen der Webseite & $\langle F010 \rangle$ \\
            \hline
                $\langle T020 \rangle$ & \Gls{Account} anlegen & $\langle F020 \rangle$ \\
            \hline
                $\langle T030 \rangle$ & Einloggen & $\langle F020 \rangle$ \\
            \hline
                $\langle T040 \rangle$ & Schüler anzeigen & $\langle F030 \rangle$\\
            \hline
                $\langle T050 \rangle$ & Neuen Schüler anlegen & $\langle F040 \rangle$\\
            \hline
                $\langle T060 \rangle$ & Schülerstatistiken ansehen & $\langle F050\rangle$ \\                
            \hline
                $\langle T070 \rangle$ & Kurse anzeigen & $\langle F060\rangle$\\
            \hline
                $\langle T080 \rangle$ & Neuen Kurs anlegen & $\langle F070\rangle$ \\
            \hline
                $\langle T090 \rangle$ & Kursstatistiken ansehen & $\langle F080\rangle$ \\
            \hline
                $\langle T100 \rangle$ & Schüler in einem Kurs anzeigen & $\langle F090 \rangle$ \\
            \hline
                $\langle T110 \rangle$ & Schüler zu einem Kurs hinzufügen & $\langle F100\rangle$ \\
            \hline
                $\langle T120 \rangle$ & Schüler aus einem Kurs entfernen & $\langle F110 \rangle$\\
            \hline
                $\langle T130 \rangle$ & Neue Sitzung erstellen & $\langle F120 \rangle$ \\
            \hline
                $\langle T140 \rangle$ & Interaktion aufzeichnen & $\langle F130 \rangle$ \\
            \hline
                $\langle T150 \rangle$ & Kategorie für Interaktion auswählen & $\langle F140 \rangle$ \\
            \hline
                $\langle T160 \rangle$ & Interaktionskarten speichern & $\langle F150 \rangle$ \\
            \hline
                $\langle T170 \rangle$ & Interaktionskarten ansehen & $\langle F160 \rangle$ \\
            \hline
                $\langle T180 \rangle$ & Sitzungsstatistiken ansehen & $\langle F170 \rangle$ \\ 
            \hline
                $\langle T190 \rangle$ & Interaktionskarten exportieren & $\langle F180 \rangle$ \\
            \hline
                $\langle T200 \rangle$ & Ausloggen & $\langle F020 \rangle$ \\
            \hline
        \end{tabular}
        \caption{Basis-Testfälle}
        \label{table:Basis-Testfaelle}
    \end{table}
        
\newpage

\paragraph{\large Erweiterte Testfälle} $~$ \\
        
    \begin{table}[h!]
            \centering
            \begin{tabular}{p{1.5cm}|p{9cm}|p{3cm}}
                    \textbf{\sffamily{Nr.}} & \textbf{\sffamily{Beschreibung}} & \textbf{\sffamily{Funktion}}\textsuperscript{\ref{table:Erweiterte Funktionen}}\\
                \hline
                \hline
                    $\langle T200 \rangle$ & Schüler teilen & $\langle F190 \rangle$\\
                \hline
                    $\langle T210 \rangle$ & Kurse teilen & $\langle F200 \rangle$\\
                \hline
                    $\langle T220 \rangle$ & Sitzordnung für einen Kurs anlegen & $\langle F210 \rangle$\\
                \hline
                    $\langle T230 \rangle$ & Unterrichtsfach zu einem Kurs hinzufügen & $\langle F220 \rangle$\\
                \hline
                    $\langle T240 \rangle$ & Interaktion rückgängig machen mit \Gls{undo}  &  $\langle F230 \rangle$\\
                \hline
                    $\langle T250 \rangle$ & Qualität bei Interaktion angeben &  $\langle F240 \rangle$\\
                \hline
                    $\langle T260 \rangle$ & Eigene Kategorie für Interaktionen erstellen &  $\langle F250 \rangle$\\
                \hline
                \end{tabular}
            \caption{Erweiterte Testfälle}
            \label{table:Erweiterte Testfaelle}
        \end{table}

    \subsection{Testszenarien}
       Ein Testszenario setzt sich aus mehreren Testfällen\textsuperscript{\ref{section:testfaelle}} zusammen.
       
    \paragraph{Szenario 1 - Erster Besuch der Webseite} $~$
        
        % Liste mit Tiefe 1
        \newlist{test1}{enumerate}{1}
        % Configure the behaviour of level 1 entries
        \setlist[test1, 1]{label=\arabic{test1i}.}
        
        \begin{test1}
            \item $\langle T010 \rangle$: Aufrufen der Webseite 
            \item $\langle T020 \rangle$: \Gls{Account} anlegen 
            \item $\langle T030 \rangle$: Einloggen 
            \item $\langle T200 \rangle$: Ausloggen 
        \end{test1}
        
        \textbf{\sffamily Ziel:} Die Seite soll angezeigt werden. Nachdem der Benutzer einen \Gls{Account} angelegt hat, soll er sich mit den eingegebenen Daten anmelden und abmelden können.\\
        
    \textbf{Im Folgenden wird vorausgesetzt, dass bereits ein \Gls{Account} angelegt worden ist, die Seite aufgerufen wurde und ein User eingeloggt ist.}
    
    \newpage
    
    \paragraph{Szenario 2 - Anlegen von einem Kurs \& Schülern und Ausgabe des Kurses \&  der Schüler} $~$ 
        
        % Liste mit Tiefe 1
        \newlist{test2}{enumerate}{1}
        \setlist[test2, 1]{label=\arabic{test2i}.}
        
        \begin{test2}
            \item $\langle T080 \rangle$: Neuen Kurs anlegen
            \item $\langle T050 \rangle$: Neuen Schüler anlegen
            \item Ausführen von 2. weitere 14-mal
            \item $\langle T070 \rangle$: Angelegten Kurs anzeigen
            \item $\langle T040 \rangle$: Angelegte Schüler anzeigen
        \end{test2}
        
        \textbf{\sffamily Ziel:} Der angelegte Kurs sowie alle angelegten Schüler müssen in den jeweiligen Listen angezeigt werden.
    
    \paragraph{Szenario 3 - Hinzufügen von Schülern zu einem Kurs und Ausgabe der Schüler} $~$ 
        
        % Liste mit Tiefe 1
        \newlist{test3}{enumerate}{1}
        \setlist[test3, 1]{label=\arabic{test3i}.}
        
        \begin{test3}
            \item $\langle T080 \rangle$: Neuen Kurs anlegen
            \item $\langle T050 \rangle$: Neuen Schüler anlegen
            \item Ausführen von 2. weitere 14-mal
            \item $\langle T110 \rangle$: Schüler zu dem angelegten Kurs hinzufügen
            \item Ausführen von 4. für alle weiteren Schüler
            \item $\langle T100 \rangle$: Schülerliste im angelegten Kurs anzeigen
        \end{test3}
        
        \textbf{\sffamily Ziel:} Es müssen alle in 2. und 3. angelegten Schüler in dem Kurs sichtbar sein.
        
        \newpage
        
    \paragraph{Szenario 4 - Entfernen von Schülern und Ausgabe} $~$ 
        
        % Liste mit Tiefe 1
        \newlist{test4}{enumerate}{1}
        \setlist[test4, 1]{label=\arabic{test4i}.}
        
        \begin{test4}
            \item $\langle T080 \rangle$: Neuen Kurs anlegen
            \item $\langle T050 \rangle$: Neuen Schüler anlegen
            \item Ausführen von 2. weitere 14-mal
            \item $\langle T110 \rangle$: Schüler zu dem angelegten Kurs hinzufügen
            \item Ausführen von 4. für alle weiteren Schüler
            \item $\langle T120 \rangle$: Schüler aus dem Kurs entfernen
            \item Ausführen von 6. weitere 4-mal
            \item $\langle T100 \rangle$: Schülerliste im angelegten Kurs anzeigen
        \end{test4}
        
        \textbf{\sffamily Ziel:} Es müssen alle in 2. und 3. angelegten Schüler, außer die entfernten, in dem Kurs sichtbar sein.
    
    \paragraph{Szenario 5 - Interaktionen aufzeichnen und ansehen} $~$ 
    
        \textbf{Es wird Testszenario 3 vorausgesetzt.}
        
        % Liste mit Tiefe 1
        \newlist{test5}{enumerate}{1}
        \setlist[test5, 1]{label=\arabic{test5i}.}
        
        \begin{test5}
            \item $\langle T130 \rangle$: Neue Sitzung erstellen
            \item $\langle T140 \rangle$: Interaktion aufzeichnen
            \item $\langle T150 \rangle$: Kategorie für Interaktion auswählen
            \item Ausführen von 2. bis 3. mindestens 10-mal
            \item $\langle T160 \rangle$: Speichern der Interaktionskarte
            \item $\langle T170 \rangle$: Anzeigen der Interaktionskarte
            \item $\langle T180 \rangle$: Sitzungsstatistiken ansehen
            \item $\langle T190 \rangle$: Interaktionskarte exportieren
        \end{test5}
        
        \textbf{\sffamily Ziel:} Alle Interaktionen zwischen den betreffenden Personen und die dazugehörenden Statistiken müssen angezeigt werden. Die Interaktionskarte wird erfolgreich exportiert.

    \paragraph{Szenario 6 - Kurs- und Schülerstatistiken ansehen} $~$ 
    
        \textbf{Es wird Testszenario 5 vorausgesetzt.}
        
        % Liste mit Tiefe 1
        \newlist{test6}{enumerate}{1}
        \setlist[test6, 1]{label=\arabic{test6i}.}
        
        \begin{test6}
            \item $\langle T070 \rangle$: Kurse anzeigen
            \item $\langle T090 \rangle$: Kurs auswählen und Kursstatistiken ansehen
            \item $\langle T100 \rangle$: Schülerliste in dem Kurs anzeigen
            \item $\langle T060 \rangle$: Schülerstatistiken ansehen
        \end{test6}
        
        \textbf{\sffamily Ziel:} Es müssen alle Statistiken zu dem Kurs und dem ausgewählten Schüler angezeigt werden.\\

    \textbf{Die folgenden Szenarien enthalten erweiterte Funktionen.}

    \paragraph{Szenario 7 - Kurs teilen} $~$ 
        
        % Liste mit Tiefe 1
        \newlist{test7}{enumerate}{1}
        \setlist[test7, 1]{label=\arabic{test7i}.}
        
        \begin{test7}
            \item $\langle T080 \rangle$: Neuen Kurs anlegen
            \item $\langle T050 \rangle$: Neuen Schüler anlegen
            \item Ausführen von 2. weitere 14-mal
            \item $\langle T110 \rangle$: Schüler zu dem angelegten Kurs hinzufügen
            \item Ausführen von 4. für alle weiteren Schüler
            \item $\langle T210 \rangle$: Angelegten Kurs mit anderem \Gls{Account} teilen
        \end{test7}
        
        \textbf{\sffamily Ziel:} Der angelegte Kurs muss in der Kursbibliothek des anderen \Gls{Account}s mit allen Schülern des Kurses angezeigt werden.

    \newpage
        
    \paragraph{Szenario 8 - Sitzordnung für Kurs erstellen} $~$ 

        \textbf{Die Nutzung der \Gls{Desktop}-Version wird vorausgesetzt.}
        
        % Liste mit Tiefe 1
        \newlist{test8}{enumerate}{1}
        \setlist[test8, 1]{label=\arabic{test8i}.}
        
        \begin{test8}
            \item $\langle T080 \rangle$: Neuen Kurs anlegen
            \item $\langle T050 \rangle$: Neuen Schüler anlegen
            \item Ausführen von 2. weitere 14-mal
            \item $\langle T110 \rangle$: Schüler zu dem angelegten Kurs hinzufügen
            \item Ausführen von 4. für alle weiteren Schüler
            \item $\langle T220 \rangle$: Sitzordnung für den Kurs anlegen
            \item $\langle T130 \rangle$: Neue Sitzung erstellen
        \end{test8}
        
        \textbf{\sffamily Ziel:} Die für den Kurs angelegte Sitzordnung muss in der Sitzung angezeigt werden.

    \paragraph{Szenario 9 - Interaktion rückgängig machen} $~$ 

        \textbf{Es wird Testszenario 3 vorausgesetzt.}
        
        % Liste mit Tiefe 1
        \newlist{test9}{enumerate}{1}
        \setlist[test9, 1]{label=\arabic{test9i}.}
        
        \begin{test9}
            \item $\langle T130 \rangle$: Neue Sitzung erstellen
            \item $\langle T140 \rangle$: Interaktion aufzeichnen
            \item $\langle T150 \rangle$: Kategorie für Interaktion auswählen
            \item Ausführen von 2. bis 3. mindestens 10-mal
            \item $\langle T240 \rangle$: Letzte Interaktion rückgängig machen
            \item $\langle T160 \rangle$: Speichern der Interaktionskarte
            \item $\langle T170 \rangle$: Anzeigen der Interaktionskarte
        \end{test9}
        
        \textbf{\sffamily Ziel:} Die mit \gls{undo} rückgängig gemachte Interaktion darf nicht angezeigt werden. 

    \newpage

    \paragraph{Szenario 10 - Eigene Interaktionskategorie erstellen} $~$ 

        \textbf{Es wird Testszenario 3 vorausgesetzt.}
        
        % Liste mit Tiefe 1
        \newlist{test10}{enumerate}{1}
        \setlist[test10, 1]{label=\arabic{test10i}.}
        
        \begin{test10}
            \item $\langle T130 \rangle$: Neue Sitzung erstellen
            \item $\langle T140 \rangle$: Interaktion aufzeichnen
            \item $\langle T260 \rangle$: Eigene Kategorie erstellen
            \item $\langle T140 \rangle$: Neue Interaktion aufzeichnen
            \item $\langle T150 \rangle$: Kategorie für Interaktion auswählen
            \item Ausführen von 4. bis 5. mindestens 10-mal
            \item $\langle T160 \rangle$: Speichern der Interaktionskarte
            \item $\langle T170 \rangle$: Anzeigen der Interaktionskarte
            \item $\langle T180 \rangle$: Sitzungsstatistiken ansehen
        \end{test10}
        
        \textbf{\sffamily Ziel:} Die neu erstellte Kategorie muss in 5. und in der Sitzungsstatistik angezeigt werden.
        
    
    
   

\printnoidxglossaries

\end{document}
